\renewcommand{\summarizedlecture}{1 }

%
%
%

\begin{frame}{Lecture \summarizedlecture - \lecturesummarytitle}

\begin{itemize}
{\small
\item {\bf Electric charge}
  \begin{itemize}
  {\small
    \item The source of electric phenomena
    \item An intrinsic property of matter
    \item Comes in two varieties (positive and negative)
    \item An algebraic quantity (+q + (-q) = 0)
    \item It is quantised
    \item It is conserved (globally and locally)
    \item SI unit: Coulomb (C) [= 1 A $\cdot$ 1 s]
  }
  \end{itemize}

\vspace{0.2cm}

\item {\bf Coulomb's law}
  \begin{itemize}
  {\small
     \item Describes the force between two point charges
     \item The force $\vec{F}_{12}$ exerted on test charge 1 by charge 2 is:
     \begin{equation*}
       \vec{F}_{12} = \frac{1}{4\pi\epsilon_0} \frac{q_1 q_2}{|\vec{r}_{1}-\vec{r}_{2}|^{2}} \hat{r}_{12}
       \;\;\;
       or
       \;\;\;
       \vec{F}_{12} = \frac{1}{4\pi\epsilon_0} \frac{q_1 q_2}{|\vec{r}_{1}-\vec{r}_{2}|^{3}} (\vec{r}_{1}-\vec{r}_{2})
     \end{equation*}
  }
  \end{itemize}

}
\end{itemize}

\end{frame}


%
%
%

\begin{frame}{Lecture \summarizedlecture - \lecturesummarytitle (cont'd)}

\begin{itemize}
{\small

\item {\bf Superposition principle}
  \begin{itemize}
  {\small
     \item Allows the calculation of the total force on a charge Q
           from an array of other charges $q_1$, $q_2$, ..., $q_n$
      \begin{equation*}
       \vec{F}_{Q} = \sum_{i=1}^{n} \frac{1}{4\pi\epsilon_0}
          \frac{Q q_i}{|\vec{r}_{Q}-\vec{r}_{q_{i}}|^{3}} (\vec{r}_{Q}-\vec{r}_{q_{i}})
       \end{equation*}
     \item Total force is the vector sum of forces.
     \item Not a logical necessity: An experimental fact!
  }
  \end{itemize}

\vspace{0.2cm}

\item {\bf Continuous distributions of charge}
  \begin{itemize}
  {\small
    \item Made the leap from discrete to continuous charge distributions described by a charge density
    \item Reformulated Coulomb's law for continuous charge distributions
     \begin{equation*}
        \vec{F}_{Q} = \frac{Q}{4\pi\epsilon_0} \int_{\tau}
           d\tau^{\prime} \frac{\rho({\pvec{r}'})}{|\vec{r}-\pvec{r}'|^{3}} (\vec{r}-\pvec{r}')
     \end{equation*}
  }
  \end{itemize}

}
\end{itemize}

\end{frame}

%
%
%

\begin{frame}{Lecture \summarizedlecture - \lecturesummarytitle (cont'd)}

\begin{itemize}
{\small

\item {\bf Electric field}
  \begin{itemize}
  {\small
     \item A more fundamental way to think about electric forces in terms of a field that permeates space.
     \item Defined the electric field $\vec{E}$
           as the force exerted on a test charge Q, placed in position $\vec{r}$, per unit charge.
      \begin{equation*}
        \vec{E}(\vec{r}) = \frac{\vec{F}_Q(\vec{r})}{Q}
      \end{equation*}
  }
  \end{itemize}

\item {\bf Visualizing the electric field - field lines}
  \begin{itemize}
  {\small
     \item Field direction indicated by the direction of field lines
     \item Field strength indicated by the density of field lines
     \item Force tangential to field lines
     \item Field lines start from positive charges and end on negative ones:
           Positive charges are "sources", negative charges are "sinks"
     \item Field lines can not cross
  }
  \end{itemize}
}
\end{itemize}

\end{frame}
