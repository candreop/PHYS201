
\renewcommand{\summarizedlecture}{5 }

%
%
%

\begin{frame}{Lecture \summarizedlecture - \lecturesummarytitle}

\begin{itemize}

\item
An {\bf electric current is a flow of electric charge.}
It is represented by the amount of charge passing though per unit time.
\begin{equation*}
  I = \frac{dQ}{dt}
\end{equation*}
In SI, the unit of the electric current is the {\bf Ampere (A)}.

\item
The current density $\vec{j}$ is the {\bf current per unit area  of cross-section}:
\begin{equation*}
  \vec{j} = n q \vec{u}_{d}
\end{equation*}
where n is the charge carrier density and $\vec{u}_{d}$ their average velocity.

\item
In general:
\begin{equation*}
  \vec{j} = \sigma \vec{E}
\end{equation*}
where $\sigma$ is the {\bf conductivity} of the material (SI unit: $1/(\Omega \cdot m)$).
The inverse quantity $\rho = 1/\sigma$ is called {\bf resistivity}.

\end{itemize}

\end{frame}

%
%
%
\begin{frame}{Lecture \summarizedlecture - \lecturesummarytitle (cont'd)}

\begin{itemize}

\item Magnetic and electric phenomena have a common origin.\\
          Remember the empirical evidence:
         \begin{itemize}
            \item Electric currents generate magnetic fields!
            \item Moving magnetic fields generate electric currents!
            \item There are magnetic forces between electric currents!
          \end{itemize}

\item The magnetic field (a vector field) is the magnetic effect of electric currents and magnetic materials (SI unit:  {\bf Tesla (T)})

\item The magnetic force on an electric charge q moving with velocity $\vec{u}$ in a magnetic field $\vec{B}$ is given by:
          $\vec{F} = q \vec{u} \times \vec{B}$

\item Consequently, the magnetic force on a current is
          $\vec{F} = I \int_{L} d\vec{\ell} \times \vec{B}$

\item Magnetic forces do no work on electric charges.

\item In the presence of both a magnetic field $\vec{B}$ and an electric field $\vec{E}$,
          the total (so-called Lorentz) force on charge q is:
          $\vec{F} = q \Big( \vec{E} + \vec{u} \times \vec{B} \Big)$

\end{itemize}

\end{frame}

%
%
%
\begin{frame}{Lecture \summarizedlecture - \lecturesummarytitle (cont'd)}

\begin{itemize}

\item Biot-Savart law (expresses $\vec{B}$ in terms of the current I):
     \begin{equation*}
            \vec{B} = \int_{L} d\vec{B}
                        = \frac{\mu_0I}{4\pi} \int_{L} \frac{d\vec{\ell} \times \vec{r}}{r^3}
     \end{equation*}
          where the integral is over the elements $d\vec{\ell}$ along the conductor, and $\vec{r}$
          is the distance from $d\vec{\ell}$ to the point where we want to know the field.

\vspace{0.2cm}

\item Biot-Savart in action: Magnetic field around a wire with current I:
     \begin{equation*}
           \vec{B}(\vec{r}) = \frac{\mu_0I}{2\pi \rho} \hat\phi
     \end{equation*}
         where $\rho$ is the distance from the wire and $\hat\phi$ the azimuthal unit vector.
\end{itemize}

\end{frame}
