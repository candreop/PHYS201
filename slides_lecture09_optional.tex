
\begin{frame}[plain,c]
\begin{center}
{\Huge \bf Optional reading for Lecture \thislecture}
\end{center}
\end{frame}

%
%
%
%

\begin{frame}{Energy carried by electromagnetic waves}

We will see that the  {\em energy carried away from the volume $\tau$ by an e/m wave
through the surface S} is given by:
\begin{equation*}
     \oint_{S} d\vec{S} \frac{1}{\mu_0} \Big( \vec{E} \times \vec{B} \Big)
\end{equation*}

\vspace{0.3cm}

Let's assume that I have an electric field $\vec{E}$ and a magnetic field $\vec{B}$ within some volume $\tau$.
A charge q moves within that volume with velocity $\vec{u}$ and it is displaced by a short distance $d\vec{\ell}$.
\begin{itemize}
   \item There is work done by the e/m force.
   \item The energy has to come from somewhere.
       \begin{itemize}
            \item Where from? Well, there is the energy
                      $\displaystyle \frac{1}{2} \int_{\tau} d\tau \Big( \epsilon_0 |\vec{E}|^2 + \frac{1}{\mu_0} |\vec{B}|^2 \Big)$
                      stored in the e/m field.
             \item So, if there is work done on the charge, the energy stored in the e/m field must be decreased.
       \end{itemize}
\end{itemize}

\end{frame}

%
%
%
%

\begin{frame}{Energy carried by electromagnetic waves}

The work done by the Lorentz force is:
\begin{equation*}
  dW = \vec{F} \cdot d\vec{\ell} = q \Big( \vec{E} + \vec{u} \times \vec{B} \Big) \cdot \vec{u} dt
\end{equation*}

We have already seen that magnetic forces do no work:
The $\vec{u} \times \vec{B}$ vector is parallel to $\vec{u}$ so the dot product
$\Big( \vec{u} \times \vec{B} \Big) \cdot \vec{u}$ is 0.

So the work done can be written just as
\begin{equation*}
  dW = \vec{F} \cdot d\vec{\ell} = q \vec{E} \cdot \vec{u} dt
\end{equation*}

To move from the discrete to the continuous case, take q to be the amount of charge contained
within a volume $d\tau$ within a region characterised by charge density $\rho$.
Then
\begin{equation*}
  dW = q \vec{E} \cdot \vec{u} dt \rightarrow \Big( \rho d\tau \Big) \vec{E} \cdot \vec{u} dt =
           dt d\tau \vec{E} \cdot \Big( \rho \vec{u} \Big)
\end{equation*}


\end{frame}


%
%
%
%

\begin{frame}{Energy carried by electromagnetic waves}

The work done per unit time, in the infinitesimal volume $d\tau$ is:
\begin{equation*}
  \frac{dW}{dt} =  d\tau \vec{E}  \cdot \Big( \rho \vec{u} \Big)
\end{equation*}

As we have seen, the product $\rho \vec{u}$ is the current density $\vec{j}$. Therefore, the work done per unit time and per unit volume can be written as:
\begin{equation*}
  \frac{dW}{dt} =  d\tau \vec{E}  \cdot \vec{j}
\end{equation*}

The total work (integrated over all space) per unit time is:
\begin{equation*}
  \frac{dW}{dt} =  \int_{\tau} d\tau \vec{E} \cdot \vec{j}
\end{equation*}

To proceed, we need to examine the dot product $\vec{E} \cdot \vec{j}$ of the electric field $\vec{E}$ and the current density $\vec{j}$.

\end{frame}

%
%
%
%

\begin{frame}{Energy carried by electromagnetic waves}

Starting from Ampere's law, we can write $\vec{j}$ in terms of
$\vec{\nabla} \times \vec{B}$ and $\frac{\partial \vec{E}}{\partial t}$:
\begin{equation*}
  \vec{\nabla} \times \vec{B} = \mu_0 \Big( \vec{j} + \epsilon_0 \frac{\partial \vec{E}}{\partial t} \Big) \Rightarrow
  \vec{j} = \frac{1}{\mu_0}  \vec{\nabla} \times \vec{B} - \epsilon_0 \frac{\partial \vec{E}}{\partial t}
\end{equation*}

Therefore:
\begin{equation*}
 \vec{E} \cdot \vec{j} =
   \vec{E} \cdot \Big( \frac{1}{\mu_0}  \vec{\nabla} \times \vec{B} - \epsilon_0 \frac{\partial \vec{E}}{\partial t} \Big) =
   \frac{1}{\mu_0}  \vec{E} \cdot \Big( \vec{\nabla} \times \vec{B} \Big) - \epsilon_0 \vec{E} \cdot \frac{\partial \vec{E}}{\partial t}
\end{equation*}

The cross-product term can be written as:
\begin{equation*}
 \vec{E} \cdot \Big( \vec{\nabla} \times \vec{B} \Big) =
 - \vec{\nabla} \Big( \vec{E} \times \vec{B} \Big) - \vec{B} \cdot \frac{\partial \vec{B}}{\partial t}
\end{equation*}
Try to prove it at home! (Details on the next slide.)

\end{frame}

%
%
%
%

\begin{frame}{Further details (skip on a first read)}

Starting from $\vec{\nabla} \Big( \vec{E} \times \vec{B} \Big)$, we can write using the {\em CAB-BAC} identity:
\begin{equation*}
 \vec{\nabla} \Big( \vec{E} \times \vec{B} \Big) = \vec{B} \Big( \vec{\nabla} \times \vec{E} \Big) - \vec{E}  \Big( \vec{\nabla} \times \vec{B} \Big)
\end{equation*}

The $\vec{E}  \Big( \vec{\nabla} \times \vec{B} \Big)$ term is what appears in the $\vec{E} \vec{j}$ dot product. Solving for it:
\begin{equation*}
 \vec{E}  \Big( \vec{\nabla} \times \vec{B} \Big) = - \vec{\nabla} \Big( \vec{E} \times \vec{B} \Big) + \vec{B} \Big( \vec{\nabla} \times \vec{E} \Big)
\end{equation*}

From Faraday's law:
\begin{equation*}
  \vec{\nabla} \times \vec{E} = - \frac{\partial \vec{B}}{\partial t}
\end{equation*}

Therefore:
\begin{equation*}
 \vec{E}  \Big( \vec{\nabla} \times \vec{B} \Big) = - \vec{\nabla} \Big( \vec{E} \times \vec{B} \Big) - \vec{B} \frac{\partial \vec{B}}{\partial t}
\end{equation*}

\end{frame}

%
%
%
%

\begin{frame}{Energy carried by electromagnetic waves}


\begin{equation*}
 \vec{E} \cdot \vec{j} =
    - \frac{1}{\mu_0} \vec{\nabla} \Big( \vec{E} \times \vec{B} \Big)
    - \frac{1}{\mu_0} \vec{B} \cdot \frac{\partial \vec{B}}{\partial t} - \epsilon_0 \vec{E} \cdot \frac{\partial \vec{E}}{\partial t}
\end{equation*}

For a vector field $\vec{F}$, we can easily see that:
\begin{equation*}
  \vec{F} \cdot \frac{\partial \vec{F}}{\partial t} = \frac{1}{2} \frac{\partial \vec{F}^2}{\partial t}
\end{equation*}

and, therefore:
\begin{equation*}
 \vec{E} \cdot \vec{j} =
    - \frac{1}{\mu_0} \vec{\nabla} \Big( \vec{E} \times \vec{B} \Big)
    - \frac{1}{2\mu_0} \frac{\partial \vec{B}^2}{\partial t} - \frac{\epsilon_0}{2} \frac{\partial \vec{E}^2}{\partial t} \Rightarrow
\end{equation*}

\begin{equation*}
 \vec{E} \cdot \vec{j} =
    - \frac{1}{\mu_0} \vec{\nabla} \Big( \vec{E} \times \vec{B} \Big)
    - \frac{\partial}{\partial t} \Big( \frac{1}{2\mu_0} \vec{B}^2 + \frac{\epsilon_0}{2} \vec{E}^2 \Big)
\end{equation*}

\end{frame}


%
%
%
%

\begin{frame}{Energy carried by electromagnetic waves}

The total work done per unit time is given, as we have seen, by integrating
$\vec{E} \cdot \vec{j}$ over the whole volume $\tau$:
\begin{equation*}
  \frac{dW}{dt} =  \int_{\tau} d\tau \vec{E} \cdot \vec{j} =
     - \int_{\tau} d\tau \frac{1}{\mu_0} \vec{\nabla} \Big( \vec{E} \times \vec{B} \Big)
     - \frac{\partial}{\partial t} \int_{\tau} d\tau  \Big( \frac{1}{2\mu_0} \vec{B}^2 + \frac{\epsilon_0}{2} \vec{E}^2 \Big)
\end{equation*}

\vspace{0.2cm}

Using Gauss' theorem, the volume integral of the divergence of a vector field can be replaced by the
surface integral of that field:
\begin{equation*}
  \frac{dW}{dt} =
     - \oint_{S} d\vec{S} \frac{1}{\mu_0} \Big( \vec{E} \times \vec{B} \Big)
     - \frac{\partial}{\partial t} \int_{\tau} d\tau  \Big( \frac{1}{2\mu_0} \vec{B}^2 + \frac{\epsilon_0}{2} \vec{E}^2 \Big)
\end{equation*}

As anticipated, the above tells you that the {\bf work done on the charge q} is, indeed, related with the
{\bf decrease of the energy stored in the e/m field} (2nd term on right-hand side).

\end{frame}


%
%
%
%

\begin{frame}{Energy carried by electromagnetic waves}

But there is another term on the right-hand side of:
\begin{equation*}
  \frac{dW}{dt} =
     - \oint_{S} d\vec{S} \frac{1}{\mu_0} \Big( \vec{E} \times \vec{B} \Big)
     - \frac{\partial}{\partial t} \int_{\tau} d\tau  \Big( \frac{1}{2\mu_0} \vec{B}^2 + \frac{\epsilon_0}{2} \vec{E}^2 \Big)
\end{equation*}

It is the surface term:
\begin{equation*}
     \oint_{S} d\vec{S} \frac{1}{\mu_0} \Big( \vec{E} \times \vec{B} \Big)
\end{equation*}
What does that term correspond to?\\

\vspace{0.2cm}

This term describes the {\em energy carried away from the volume $\tau$,
through its surface S, by the electromagnetic fields}.\\

\vspace{0.2cm}

So the energy stored in the e/m field in volume $\tau$ is decreased, because of the
work done on the charge, but also because energy is flowing out.\\

\vspace{0.2cm}

This is the {\bf Poynting theorem}.

\end{frame}
