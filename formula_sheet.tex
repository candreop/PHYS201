\documentclass[english,11pt]{article}

\usepackage{abstract}
\renewcommand{\abstractname}{}    % clear the title
\renewcommand{\absnamepos}{empty} % originally center

\usepackage[affil-it]{authblk}

\usepackage{amsmath}
\usepackage{graphicx}
\usepackage{color}
\usepackage{mathtools} % \xRightarrow{text} etc
\usepackage{cancel}    % \cancel, \cancelto{}
\usepackage{float}     % option H in tables etc
\usepackage{dcolumn}   % line up tabular columns on the decimal place with D{.}{.}{-1}
\usepackage{rotating}
\usepackage{multirow}  % \multirow
\usepackage{hhline}    % \hhline{~--~}
\usepackage{extarrows} % \xlongequal{}

\usepackage[T1]{fontenc}
\usepackage[latin1]{inputenc}

% primed vector with more space between the prime and the arrow
\newcommand{\pvec}[1]{\vec{#1}\mkern2mu\vphantom{#1}}

\setlength{\oddsidemargin}{-6mm}
\setlength{\evensidemargin}{-6mm}
\setlength{\textwidth}{6.5in}
\setlength{\textheight}{9.0in}
\setlength{\voffset}{-0.75in}

\begin{document}

\title{
  {\bf PHYS201 (Electromagnetism)}\\
  Formula Sheet
}

\author{
  Professor Costas Andreopoulos
  \thanks {
    Electronic address: \texttt{constantinos.andreopoulos@cern.ch}
  }
}

\affil
{
  University of Liverpool \& U.K. Research and Innovation, \\
  Science and Technology Facilities Council, Rutherford Appleton Laboratory
}

\date{}

\maketitle

\begin{abstract}
 % Solve the following problems during the workshop ({\bf Thursday 4 October 2018}). \\
 % \input{instructions.tex}
\end{abstract}

\subsubsection*{\bf Coulomb's law - Force between two discrete charges $q_1$, $q_2$}

  \begin{itemize}
  {\small
     \item The force $\vec{F}_{12}$ exerted on test charge $q_1$
           (placed in $\vec{r}_{1}$) by charge $q_2$ (placed in $\vec{r}_{2}$) is:
     \begin{equation*}
       \vec{F}_{12} = \frac{1}{4\pi\epsilon_0} \frac{q_1 q_2}{|\vec{r}_{1}-\vec{r}_{2}|^{2}} \hat{r}_{12}
       \;\;\;
       or
       \;\;\;
       \vec{F}_{12} = \frac{1}{4\pi\epsilon_0} \frac{q_1 q_2}{|\vec{r}_{1}-\vec{r}_{2}|^{3}} (\vec{r}_{1}-\vec{r}_{2})
     \end{equation*}
     In the above expression, $\hat{r}_{12}$ is a unit vector in the direction of $\vec{r}_{1}-\vec{r}_{2}$:
     \begin{equation*}
        \hat{r}_{12} = \frac{\vec{r}_{1}-\vec{r}_{2}}{|\vec{r}_{1}-\vec{r}_{2}|}
     \end{equation*}
     \item Note that the force $\vec{F}_{21}$ exerted on test charge $q_2$ by charge $q_1$ is:
     \begin{equation*}
       \vec{F}_{21} =
          \frac{1}{4\pi\epsilon_0} \frac{q_2 q_1}{|\vec{r}_{2}-\vec{r}_{1}|^{3}} (\vec{r}_{2}-\vec{r}_{1}) =
          \frac{1}{4\pi\epsilon_0} \frac{q_1 q_2}{|\vec{r}_{1}-\vec{r}_{2}|^{3}} (-\vec{r}_{1}+\vec{r}_{2}) =
        - \frac{1}{4\pi\epsilon_0} \frac{q_1 q_2}{|\vec{r}_{1}-\vec{r}_{2}|^{3}} (\vec{r}_{1}-\vec{r}_{2}) =
        - \vec{F}_{12}
    \end{equation*}
  }
  \end{itemize}

\subsubsection*{\bf Superposition principle - Force on charge Q due to an array of discrete charges}
    \begin{itemize}
    {\small
       \item Allows the calculation of the total force on a charge $Q$
             from an array of other charges $q_1$, $q_2$, ..., $q_n$
        \begin{equation*}
         \vec{F}_{Q} = \sum_{i=1}^{n} \frac{1}{4\pi\epsilon_0}
            \frac{Q q_i}{|\vec{r}_{Q}-\vec{r}_{q_{i}}|^{3}} (\vec{r}_{Q}-\vec{r}_{q_{i}})
         \end{equation*}
       \item Total force is the vector sum of forces.
    }
    \end{itemize}

\subsubsection*{\bf Generalization of Coulomb's law - Force on Q due to a continuous distribution of charge}

In general,the force on a charge $Q$, placed at position $\vec{r}$,
due to a continuous distribution of charge, is given by:
      {\small
         \begin{equation*}
            \vec{F}_{Q} = \frac{Q}{4\pi\epsilon_0} \int_{\tau}
               \frac{dq({\pvec{r}'})}{|\vec{r}-\pvec{r}'|^{3}} (\vec{r}-\pvec{r}')
         \end{equation*}
      }
where $dq$ is the infinitesimal amount of charge in the viscinity of point $\pvec{r}'$.

The continuous distribution of charge can be described by a volume charge density $\rho(\pvec{r}')$
(charge per unit volume), in which case the above expression becomes:
      {\small
         \begin{equation*}
            \vec{F}_{Q} = \frac{Q}{4\pi\epsilon_0} \int_{\tau}
               d\tau^{\prime} \frac{\rho({\pvec{r}'})}{|\vec{r}-\pvec{r}'|^{3}} (\vec{r}-\pvec{r}')
         \end{equation*}
      }
If one or two spatial dimensions of a problem can be ignored, the distribution of charge
can be described with a surface charge density $\sigma(\pvec{r}')$ (charge per unit area)
or a linear charge density $\lambda(\pvec{r}')$ (charge per unit length) respectively.
In these cases, $\vec{F}_{Q}$ is determined from the following integral over the
corresponding surface $S$ or line $L$:
{\small
   \begin{equation*}
      \vec{F}_{Q} = \frac{Q}{4\pi\epsilon_0} \int_{S}
         dS^{\prime} \frac{\sigma({\pvec{r}'})}{|\vec{r}-\pvec{r}'|^{3}} (\vec{r}-\pvec{r}')
      \;\;\;\;\; or \;\;\;\;\;
      \vec{F}_{Q} = \frac{Q}{4\pi\epsilon_0} \int_{L}
        d\ell^{\prime} \frac{\lambda({\pvec{r}'})}{|\vec{r}-\pvec{r}'|^{3}} (\vec{r}-\pvec{r}')
   \end{equation*}
}

\subsubsection*{\bf Electric field}
        \begin{itemize}
        {\small
           \item A more fundamental way to think about electric forces in terms of a field that permeates space.
           \item Defined the electric field $\vec{E}$
                 as the force exerted on a test charge Q, placed in position $\vec{r}$, per unit charge.
            \begin{equation*}
              \vec{E}(\vec{r}) = \frac{\vec{F}_Q(\vec{r})}{Q}
            \end{equation*}
        }
        \end{itemize}

\subsubsection*{\bf Electric flux}
          \begin{itemize}
          {\small
             \item The electric flux $\Phi_E$ is the number of field lines of the electric field $\vec{E}$
                   flowing through a surface S
                \begin{equation*}
                  \Phi_E = \int_{S} \vec{E} \cdot d\vec{S}
                \end{equation*}
            \item On many occassions in this lecture series, we study the flux through a closed surface S
                \begin{equation*}
                  \Phi_E = \oint_{S} \vec{E} \cdot d\vec{S}
                \end{equation*}
            \item The distinction is always clear from context. When I want to empasize the difference
                  (see section on Maxwell's equations below) I will denote the former as $\Phi^{open}_{E}$
          }
          \end{itemize}


\subsubsection*{\bf Work}

        \begin{itemize}
        {\small
          \item
          A force is said to {\bf do work} (denoted with W) if, when it is acting on a body,
          there is a {\bf displacement of the point of application in the direction of the force}.
          \item
          The work dW done by a force $\vec{F}$ displacing the point of application by $d\vec{\ell}$,
          is given by the dot product:
              \begin{equation*}
                dW = \vec{F} \cdot d\vec{\ell}
              \end{equation*}
          \item
          Notice that work is a {\bf scalar}.
          \item
          Note that {\bf a force perpendicular to the direction of motion does no work}.
          \item
          The {\bf work can be positive or negative}.
          By convention, we take work to be negative if it opposes the motion, i.e. $\theta > 90^{o}$.
          \item
          The total work along a trajectory is given by integrating (summing up) the work done for
          each infinitesimal displacement $d\vec{\ell}$:
             \begin{equation*}
                W = \int dW = \int \vec{F} \cdot d\vec{\ell}
             \end{equation*}
       }
       \end{itemize}

\subsubsection*{\bf Electrostatic potential energy}

The work $W$ done to assemble a system of charges becomes
electrostatic potential energy $U$ ($U=W$)
and it is stored in the field produced by the charges.
The formulas below show how to calculate $U$ for a:

          \begin{itemize}
          \item system of 2 charges:
           \begin{equation*}
                U = \frac{q_1 q_2}{4\pi\epsilon_0} \frac{1}{|\vec{r}_{12}|}
           \end{equation*}

          \item system of 3 charges:
           \begin{equation*}
             U = \frac{q_1 q_2}{4\pi\epsilon_0} \frac{1}{|\vec{r}_{12}|} +
                 \frac{q_1 q_3}{4\pi\epsilon_0} \frac{1}{|\vec{r}_{13}|} +
                 \frac{q_2 q_3}{4\pi\epsilon_0} \frac{1}{|\vec{r}_{23}|}
           \end{equation*}

          \item system of N charges:
           \begin{equation*}
             U = \frac{1}{2} \sum_{i,j=1;i{\ne}j}^{N} \frac{q_i q_j}{4\pi\epsilon_0|\vec{r}_{ij}|}
           \end{equation*}

          \item continuous charge distribution (with density $\rho$ over a volume $\tau$):
           \begin{equation*}
              U = \frac{1}{2} \int_{\tau} \int_{\tau^{\prime}}
                \frac{dq(\vec{r}) dq^{\prime}(\vec{r^{\prime}})}{4\pi\epsilon_0|\vec{r} - \vec{r^{\prime}}|}
                = \frac{1}{2} \int_{\tau} d\tau \int_{\tau^{\prime}} d\tau^{\prime}
                  \frac{\rho(\vec{r}) \rho(\vec{r^{\prime}})}{4\pi\epsilon_0|\vec{r} - \vec{r^{\prime}}|}
           \end{equation*}

          \end{itemize}

\subsubsection*{\bf Electrostatic potential}

The electrostatic potential $V(\vec{r})$ is a scalar field,
and it represents the amount of work $W$ required to bring a charge $Q$
in position $\vec{r}$, divided by the charge $Q$:
\begin{equation*}
     V = \frac{W}{Q}
\end{equation*}
The expresions below show how to compute $V(\vec{r})$ due to:
\begin{itemize}
\item a point charge $q$ placed at $\vec{r^{\prime}}$:
 \begin{equation*}
      V(\vec{r}) = \frac{q}{4\pi\epsilon_0} \frac{1}{|\vec{r}-\vec{r^{\prime}}|}
 \end{equation*}

\item charges $q_1$, $q_2$,..., $q_N$ placed at positions
      $\vec{r}_1$, $\vec{r}_2$,..., $\vec{r}_N$:
  \begin{equation*}
      V(\vec{r}) = \frac{1}{4\pi\epsilon_0}
            \sum_{i=1}^{N} \frac{q_i}{|\vec{r}-\vec{r}_{i}|}
  \end{equation*}

\item a continuous charge distribution represented by $\rho(r^{\prime})$ :
\begin{equation*}
   V(\vec{r}) = \frac{1}{4\pi\epsilon_0}
     \int_{\tau^\prime} \frac{dq(\vec{r}^{\;\prime})}{|\vec{r}-\vec{r}^{\;\prime}|}
     = \frac{1}{4\pi\epsilon_0}
     \int_{\tau^\prime} \frac{\rho(\vec{r}^{\;\prime})d\tau^\prime}{|\vec{r}-\vec{r}^{\;\prime}|}
\end{equation*}

\end{itemize}


\subsubsection*{\bf Definition of electric current}

An {\bf electric current is a flow of electric charge.}
It is represented by the amount of charge passing though per unit time.
\begin{equation*}
  I = \frac{dQ}{dt}
\end{equation*}

\subsubsection*{\bf Current density $\vec{j}$}

$\vec{j}$ represents the amount of current per unit area {\bf perpendicular} to the direction of the current.
\begin{equation*}
  I = \int_{S} \vec{j} \cdot d\vec{S}
\end{equation*}

\subsubsection*{\bf Microscopic view of electric current}

The current density $\vec{j}$ is given by
\begin{equation*}
  \vec{j} = n q \vec{u}_{d}
\end{equation*}
where n is the density of the carriers of charge q, and $\vec{u}_{d}$ their average (drift) velocity.

In general:
\begin{equation*}
  \vec{j} = \sigma \vec{E}
\end{equation*}
where $\sigma$ is the {\bf conductivity} of the material (SI unit: $1/(\Omega \cdot m)$).
The inverse quantity $\rho = 1/\sigma$ is called {\bf resistivity}.

\subsubsection*{\bf Continuity equation}

Expresses local conservation of charge:
\begin{equation*}
    \vec{\nabla} \cdot \vec{j} +\frac{d\rho}{dt} = 0
\end{equation*}


\subsubsection*{\bf Electric force on a charge}
\begin{equation*}
  \vec{F}_{E} = q \vec{E}
\end{equation*}

\subsubsection*{\bf Magnetic force on a (moving) charge}
\begin{equation*}
  \vec{F}_{M} = q \vec{u} \times \vec{B}
\end{equation*}

\subsubsection*{\bf Lorentz force on a charge}
The total force felt by a charged body in the presence of
both electric and magnetic fields is called the {\bf Lorentz force}.
\begin{equation*}
  \vec{F} = q \Big( \vec{E} + \vec{u} \times \vec{B} \Big)
\end{equation*}

\subsubsection*{\bf Magnetic force on a steady current}
\begin{equation*}
  \vec{F}_{M} = I \int_{L} d\vec{\ell} \times \vec{B}
\end{equation*}

\subsubsection*{\bf Generalization of magnetic force on a current}
\begin{equation*}
  \vec{F}_{M} =  \int_{\tau} \vec{j} \times \vec{B} d\tau
\end{equation*}


\subsubsection*{\bf Magnetic force between two steady currents}
At distance $\rho$, the force between two parallel straight
conductors of length L carrying current $I_1$ and $I_2$,
respectively, is:
\begin{equation*}
  F = \frac{\mu_0 I_{1} I_{2} L}{2\pi \rho}
\end{equation*}
The force is attractive if both currents flow in the same direction, and
repulsive if the two currents flow in opposite directions.

\subsubsection*{\bf Cyclotron motion}

The equation
\begin{equation*}
  m u = q B r \Rightarrow
  {\bf p = qBr}
\end{equation*}
where p is the particle momentum
is known as the {\bf cyclotron formula} and describes the motion of a particle with charge q
perpendicularly to a uniform magnetic field B.
The particle moves in a circle of radius r.\\

The period of rotation is given by:
\begin{equation*}
  T = \frac{2\pi r}{u} \xRightarrow{m u = q B r}
  T = \frac{2\pi m}{q B}
\end{equation*}

\subsubsection*{\bf Biot-Savart law}

Expresses $\vec{B}$ in terms of a steady current I:
\begin{equation*}
  \vec{B} = \int_{L} d\vec{B}
      = \frac{\mu_0I}{4\pi} \int_{L} \frac{d\vec{\ell} \times \vec{r}}{r^3}
\end{equation*}
where the integral is over the elements $d\vec{\ell}$ along the conductor, and $\vec{r}$
is the distance from $d\vec{\ell}$ to the point where we want to know the field.

\subsubsection*{\bf Generalization of Biot-Savart law}

\begin{equation*}
    \vec{B} =
      \frac{\mu_0}{4\pi} \int_{\tau^{\prime}}
         \frac{\vec{j}(\pvec{r}') \times \vec{r}}{r^{3}} d\tau^{\prime}
\end{equation*}


\subsubsection*{\bf Magnetic field around an infinitely-long straight wire with current I}
\begin{equation*}
      |\vec{B}| = \frac{\mu_0I}{2\pi \rho}
\end{equation*}
where $\rho$ is the distance from the wire.
The field is azimuthal.

\subsubsection*{\bf Magnetic field inside a solenoid with n windings per unit length, each carrying current I}
\begin{equation*}
      |\vec{B}| = \mu_0 n I
\end{equation*}
The field is in the direction of the axis of the solenoid.

\subsubsection*{\bf Magnetic field of a toroidal coil with N windings, each carrying current I}

\begin{equation*}
  |\vec{B}| = \left\{ \begin{array}{l}
         0, \text{for $r < a$,} \\
         \frac{\mu_0 N I}{2\pi r}, \text{for $a < r < b$, and} \\
         0, \text{for $r > b$} \\
    \end{array} \right.
\end{equation*}
where a (b) is the inner (outer) radius.


\subsubsection*{\bf Wave equation}

A {\bf wave equation} describes how a {\em disturbance} propagates in time.\\

Let $\phi(\vec{r}, t)$ be a function that describes that disturbance as a function of position in space and time.
Then it satisfies the following equation:
\begin{equation*}
   \vec{\nabla}^{2} \phi(\vec{r}, t) = \frac{1}{u^2} \frac{\partial^{2} \phi(\vec{r}, t)} {\partial t^{2}}
\end{equation*}
where {\bf u is the wave velocity}.\\


\subsubsection*{\bf The electric and magnetic fields (in vacuum) satisfy a wave equation}

\begin{equation*}
  \vec{\nabla}^{2} \vec{E} = \mu_0 \epsilon_0 \frac{\partial^{2} \vec{E}}{\partial t^{2}}
\end{equation*}

\begin{equation*}
  \vec{\nabla}^{2} \vec{B} = \mu_0 \epsilon_0 \frac{\partial^{2} \vec{B}}{\partial t^{2}}
\end{equation*}

\subsubsection*{\bf Speed of electromagnetic waves in vacuum}
\begin{equation*}
  c = \frac{1}{\sqrt{\mu_0 \epsilon_0}}
\end{equation*}


\subsubsection*{\bf The electric and magnetic fields (in matter) satisfy a wave equation}

\begin{equation*}
  \vec{\nabla}^{2} \vec{E} = \mu \epsilon \frac{\partial^{2} \vec{E}}{\partial t^{2}}
\end{equation*}

\begin{equation*}
  \vec{\nabla}^{2} \vec{B} = \mu \epsilon \frac{\partial^{2} \vec{B}}{\partial t^{2}}
\end{equation*}

\subsubsection*{\bf Speed of electromagnetic waves in matter}
\begin{equation*}
  u = \frac{1}{\sqrt{\mu \epsilon}} < c
\end{equation*}


\subsubsection*{\bf Index of refraction of a material}
\begin{equation*}
  n = \frac{c}{u} =
    \frac{\sqrt{\mu \epsilon}}{\sqrt{\mu_{0} \epsilon_{0}}} =
    \frac{\sqrt{\epsilon_r \epsilon_0 \mu_r
        \mu_0}}{\sqrt{\epsilon_0 \mu_0}} = \sqrt{\epsilon_r \mu_r}
\end{equation*}


\subsubsection*{\bf Energy stored in the electric field}

\begin{equation*}
   U_{E} = \frac{\epsilon_0}{2} \int_{all\;space} |\vec{E}(\vec{r})|^2  d\tau
\end{equation*}

The energy density (energy per unit volume) is:
\begin{equation*}
  u_{E} = \frac{\epsilon_0}{2} |\vec{E}(\vec{r})|^2
\end{equation*}

\subsubsection*{\bf Energy stored in the magnetic field}

\begin{equation*}
   U_{M} = \frac{1}{2\mu_{0}} \int_{all\;space} |\vec{B}(\vec{r})|^2  d\tau
\end{equation*}
The energy density (energy per unit volume) is:
\begin{equation*}
  u_{M} = \frac{1}{2\mu_{0}} |\vec{B}(\vec{r})|^2
\end{equation*}


\subsubsection*{\bf Energy stored in the electromagnetic field}

\begin{equation*}
   U_{EM} = U_{E} + U_{M} = \int_{all\;space} \Big(
     \frac{\epsilon_0}{2} |\vec{E}(\vec{r})|^2 + \frac{1}{2\mu_{0}} |\vec{B}(\vec{r})|^2
     \Big) d\tau
\end{equation*}


\subsubsection*{\bf Poynting theorem}

\begin{equation*}
  \frac{dW}{dt} =
     - \oint_{S} d\vec{S} \cdot \frac{1}{\mu_0} \Big(\vec{E} \times \vec{B} \Big)
     - \frac{\partial}{\partial t} \int_{\tau} d\tau  \Big( \frac{\epsilon_0}{2} |\vec{E}|^2 + \frac{1}{2\mu_0} |\vec{B}|^2 \Big)
\end{equation*}


\subsubsection*{\bf Poynting vector $\vec{N}$}

\begin{equation*}
  \vec{N} = \frac{1}{\mu_0} \Big( \vec{E} \times \vec{B} \Big)
\end{equation*}
It represents the {\bf energy flux density (rate energy transfer per unit area)} and it has units of $W(att)/m^2$.\\

The average power transmitted by an electromagnetic wave, is given by
the {\bf average of $\vec{N}$ over a period T}:
\begin{equation*}
  <\vec{N}> = \frac{1}{T} \int_{0}^{T} \vec{N} dt
\end{equation*}

The magnitude of the averaged Poynting vector
can be written in terms of the amplitudes of the electric field,
$E_0$, and magnetic field, $B_0$, as
\begin{equation*}
  <N> = \frac{E_0 B_0}{2 \mu_0}
\end{equation*}

Considering that $B_0 = E_0/c$, and introducing the root mean square
amplitudes $E_{rms}=E_0/\sqrt{2}$ and $B_{rms}=B_0/\sqrt{2}$,
the above can be rewritten  as
\begin{equation*}
  <N> =
  \frac{c B_0}{2 \mu_0} =
  \frac{c B_{rms}^2}{\mu_0} =
  \frac{c E_0}{2 \mu_0 c} =
  \frac{c E_{rms}^2}{\mu_0 c}
\end{equation*}


\subsubsection*{\bf Radiation pressure}

If an object is illuminated by radiation for a time interval ${\Delta}t$,
during which it absorbs energy
\begin{equation*}
  {\Delta}U = IA{\Delta}t
\end{equation*}
where I is the intensity I (power per area, or energy per time
per area) of the radiation, and A is the area of the object.
\begin{itemize}
\item
The momentum change ${\Delta}p$ of the object is given by:
\begin{equation*}
    {\Delta}p = \frac{{\Delta}U}{c}
    \;\;\; {\color{red} \small (total \; absorption)}
    \;\;\;
    {\Delta}p = \frac{2{\Delta}U}{c}
    \;\;\; {\color{red} \small (total \; reflection)}
\end{equation*}
\item
The radiation pressure $P_r$ exterted on the object is given by:
\begin{equation*}
  P_r = \frac{I}{c}
  \;\;\; {\color{red} \small (total \; absorption)}
  \;\;\;
  P_r = \frac{2I}{c}
  \;\;\; {\color{red} \small (total \; reflection)}
\end{equation*}
\end{itemize}

\subsubsection*{\bf Boundary conditions for the electric field}

\begin{equation*}
    \epsilon_1 E_1^{\perp} = \epsilon_2 E_2^{\perp} \Rightarrow
    D_1^{\perp} = D_2^{\perp}
\end{equation*}

\begin{equation*}
    E_1^{\parallel} = E_2^{\parallel}
\end{equation*}

\subsubsection*{\bf Boundary conditions for the magnetic field}

\begin{equation*}
    B_1^{\perp} = B_2^{\perp}
\end{equation*}

\begin{equation*}
    \frac{1}{\mu_1} B_1^{\parallel} = \frac{1}{\mu_2} B_2^{\parallel} \Rightarrow
    H_1^{\parallel} = H_2^{\parallel}
\end{equation*}


\subsubsection*{\bf Intensity of unpolarized light passing through a polarizing sheet}
If $I_0$ is the intensity of the unpolarized light,
the intensity $I$ of the transmitted light is:
\begin{equation*}
  I = \frac{1}{2}I_0
\end{equation*}


\subsubsection*{\bf Intensity of polarized light passing through a polarizing sheet}
If the light reaching the filter is already polarized,
the intensity $I$ of the transmitted light is:
\begin{equation*}
  I = I_0 cos^2\theta
\end{equation*}
where $\theta$ is the angle between the electric field $\vec{E}$
and the polarizing direction of the sheet.

\subsubsection*{\bf Laws of geometrical optics}
\begin{itemize}
  \item {\bf 1$^{st}$ Law}:\\
     The incident, reflected, and transmitted wave vectors form a plane (plane of incidence),
     which also includes the normal to the surface.
     \begin{equation*}
          k_I sin\theta_I = k_R sin\theta_R = k_T sin\theta_T
      \end{equation*}
  \item {\bf 2$^{nd}$ Law} ({\bf Law of reflection}):\\
     The angle of incidence is equal to the angle of reflection.
     \begin{equation*}
          \theta_I = \theta_R
      \end{equation*}
  \item {\bf 3$^{rd}$ Law} ({\bf Law of refraction} or {\bf Snell's law}):\\
     \begin{equation*}
          \frac{sin\theta_T}{sin\theta_I} = \frac{n_1}{n_2}
      \end{equation*}
\end{itemize}

\subsubsection*{\bf Total internal reflection}
As the angle of incidence increases, the angle of refraction increases.
For some critical value $\theta_c$, the angle of refraction becomes 90$^o$.\\
\vspace{0.1cm}
For angles of incidence larger than $\theta_c$, such as for rays f and
g below, there is no refracted ray and all the light is reflected;
this effect is called {\bf total internal reflection}.\\
\begin{equation*}
       n_1 sin\theta_c = n_2 sin90^o \Rightarrow
       \theta_c = sin^{-1} \frac{n_2}{n_1} \;\;\;\;\;(n_2 < n_1)
\end{equation*}

\subsubsection*{\bf Polarization by reflection / Brewster angle}
When the light is incident at a particular incident angle, called {\bf Brewster angle}
given by:
\begin{equation*}
     tan\theta_B \approx \frac{n_2}{n_1}
 \end{equation*}
the reflected light has only perpendicular components: It is
{\bf fully polarized} perpendicular to the plane of incidence.
For light incident at that angle, the reflected and refracted rays are perpendicular to each other.

\subsubsection*{\bf Maxwell's equations - Static case (time-independent fields) in vacuum}

{\small
\begin{center}
{
  \begin{table}[H]
    \begin{tabular}{|l|c|c|}
      \hline
      %   \multicolumn{3}{|l|} {
      %     {\color{magenta}
      %      {\bf Static case (time-independent fields) in vacuum}
      %     }
      %   }\\
      % \hline
      {\bf Gauss's law} &
        $\Phi_{E} = \displaystyle \oint_{S} \vec{E} \cdot d\vec{S} = \frac{1}{\epsilon_0} \int_{\tau(S)} \rho d\tau = \frac{Q}{\epsilon_0}$ &
        $\displaystyle \vec{\nabla} \cdot \vec{E} = \frac{\rho}{\epsilon_0}$ \\

      {\bf Circuital law} &
        $\displaystyle \oint_{L} \vec{E} \cdot d\vec{\ell} = 0$ &
        $\displaystyle \vec{\nabla} \times \vec{E} = 0$ \\

      {\bf Gauss's law} (magn) &
        $\Phi_{B} = \displaystyle  \oint_{S} \vec{B} \cdot d\vec{S} = 0$ &
        $\displaystyle  \vec{\nabla} \cdot \vec{B} = 0$ \\

      {\bf Ampere's law} &
        $\displaystyle \oint_{L} \vec{B} \cdot d\vec{\ell} = \mu_{0} \int_{S(L)} \vec{j} \cdot d\vec{S} = \mu_{0} I$ &
        $\displaystyle \vec{\nabla} \times \vec{B} = \mu_{0} \vec{j}$ \\
      \hline
    \end{tabular}
  \end{table}
}
\end{center}
}%small

\subsubsection*{\bf Static fields in matter -
 Polarization (magnetization) and induced charges (currents)}

In materials, macroscopic polarization induces surface and volume charges that
contribute to the total electic field and need to be taken into account.
Similarly, macroscopic magnetization induces surface and volume currents that
contribute to the total magnetic field and need to be taken into account.
The relevant expressions for the induced charges and currents are given below.

{
% \setlength{\extrarowheight}{8pt}
% \setlength{\arraycolsep}{5pt}

\begin{center}
  \begin{table}[H]
    \begin{tabular}{c|c||c|c}
      \hline
      \multicolumn{2}{c||}{\bf Electrostatics} &
      \multicolumn{2}{c}  {\bf Magnetostatics} \\
      \hline
         {\scriptsize electric dipole moment} &
         $\vec{p} = q \vec{d}$ &
         $\vec{m} = I \vec{S}$ &
         {\scriptsize magnetic dipole moment} \\
      \hline
         {\scriptsize torque within $\vec{E}$ field} &
         $\vec{T} = \vec{p} \times \vec{E}$ &
         $\vec{T} = \vec{m} \times \vec{B}$ &
         {\scriptsize torque within a $\vec{B}$ field} \\
      \hline
         {\scriptsize polarization} &
         $\vec{P}  = \frac{(e.d.m)}{volume}$ &
         $\vec{M} = \frac{(m.d.m)}{volume}$ &
         {\scriptsize magnetization} \\
      \hline
         {\scriptsize surface charge density} &
         $\sigma_{P} = \vec{P} \cdot \hat{n}$ &
         $j_{m}^{surf} = \vec{M} \times \hat{n}$ &
         {\scriptsize surface current density} \\
      \hline
         {\scriptsize volume charge density} &
         $\rho_{P} = - \vec{\nabla} \cdot \vec{P}$ &
         $j_{m}^{vol} = \vec{\nabla} \times \vec{M}$ &
         {\scriptsize volume current density} \\
      % \hline
      %    {\scriptsize electric displacement} &
      %    $\vec{D} = \epsilon_0 \vec{E} + \vec{P}$ &
      %    $\vec{H} = \frac{\vec{B}}{\mu_0} - \vec{M}$ &
      %    {\scriptsize magnetizing field} \\
      % \hline
      %    \multirow{2}{*}{\scriptsize Gauss' law in materials} &
      %    $\vec{\nabla} \cdot \vec{D} = \rho_{f}$ &
      %    $\vec{\nabla} \times \vec{H} = \vec{j}_{f}$ &
      %    \multirow{2}{*}{\scriptsize Ampere's law in materials} \\
      % \hhline{~--~}
      %    &
      %    $\oint_{S} \vec{D} \cdot d\vec{S} = Q_{f}$ &
      %    $\oint_{L} \vec{H} \cdot d\vec{\ell} = I_{f}$ &
      %    \\
      \hline
    \end{tabular}
  \end{table}
\end{center}
}

\subsubsection*{\bf Auxiliary fields $\vec{D}$ and $\vec{H}$}

The introduction of auxiliary fields $\vec{D}$ (electric displacement)
and $\vec{H}$ (H field, or magnetizing field), given below, simplifies then
description of electric and magnetic phenomena in materials, since they allow
one to write Maxwell's equation only in terms of free charges and currents.

Electric displacement $\vec{D}$:
\begin{equation*}
   \vec{D} = \epsilon_0 \vec{E} + \vec{P}
 \end{equation*}

Magnetizing field $\vec{H}$:
\begin{equation*}
   \vec{H} = \frac{\vec{B}}{\mu_0} - \vec{M}
\end{equation*}


\subsubsection*{\bf Relationship between polarization (magnetization)
  and the electric (magnetic) field}

The polarisation vector $\vec{P}$ can be expressed in terms of $\vec{E}$:
\begin{equation*}
  \vec{P} = \chi_e \epsilon_0 \vec{E}
\end{equation*}
where $\chi_e$ is the so-called {\bf electric susceptibility} (dimensionless).\\


If the analogy between electrostatics and magnetostatics was exact,
we would write $\vec{M}$ in terms of $\vec{B}$.
However, this is where the analogy breaks.
Instead we express $\vec{M}$ in terms of $\vec{H}$:
\begin{equation*}
  \vec{M} = \chi_{m} \vec{H}
\end{equation*}
where  $\chi_m$ is the {\bf magnetic susceptibility}.

\subsubsection*{\bf Polarization and magnetization for linear materials}

For linear dielectrics (and low intensity fields) $\chi_e$ is a
constant that does not depend on $\vec{E}$.
Therefore, the displacement vector $\vec{D}$ can be written as:
\begin{equation*}
  \vec{D} = \epsilon_0 \vec{E} + \vec{P} =
  \epsilon_0 \vec{E} + \chi_e \epsilon_0 \vec{E} =
  (1+\chi_e) \epsilon_0 \vec{E} = \rho_{f} =
  \epsilon_r \epsilon_0 \vec{E} = \rho_{f} =
  \epsilon \vec{E}
\end{equation*}
where the factor $\epsilon_r = 1+\chi_e$
is the {\bf relative permittivity} or {\bf dielectric constant} (dimensionless) and
$\epsilon = \epsilon_r  \epsilon_0$ is the {\bf permittivity} of the dielectric\\


Similarly, for linear materials,
$\chi_m$ is a constant independent of the value of $\vec{H}$.
Expressing $\vec{B}$ in terms of $\vec{H}$:
\begin{equation*}
  \vec{H} = \frac{\vec{B}}{\mu_0} - \vec{M} \Rightarrow
  \vec{B} = \mu_0 \Big( \vec{H} + \vec{M} \Big) \xRightarrow{\vec{M} = \chi_{m} \vec{H}}
  \vec{B} = \Big(1 + \chi_{m} \Big) \mu_0  \vec{H} \Rightarrow
\end{equation*}
\begin{equation*}
  \vec{B} = \mu_r \mu_0 \vec{H} \Rightarrow
  \vec{B} = \mu \vec{H}
\end{equation*}
where $\mu_r = 1+\chi_{\mu}$
is the {\bf relative permeability} (dimensionless) and
$\mu = \mu_r  \mu_0$ is the {\bf permeability} of the material


\subsubsection*{\bf Maxwell's equations - Static case (time-independent fields) in matter}

{\small
\begin{center}
  \begin{table}[H]
    \begin{tabular}{|l|c|c|}
      % \hline
      %   \multicolumn{3}{|l|} {
      %     {\color{magenta}
      %      {\bf Static case (time-independent fields) in matter}
      %     }
      %   }\\
      \hline
      {\bf Gauss's law} &
        $\Phi_{D} = \displaystyle \oint_{S} \vec{D} \cdot d\vec{S} =  \int_{\tau(S)} \rho_{free} d\tau = Q_{free}$ &
        $\displaystyle \vec{\nabla} \cdot \vec{D} = \rho_{free}$ \\

      {\bf Circuital law} &
        $\displaystyle \oint_{L} \vec{E} \cdot d\vec{\ell} = 0$ &
        $\displaystyle \vec{\nabla} \times \vec{E} = 0$ \\

      {\bf Gauss's law} (magn.) &
        $\Phi_{B} = \displaystyle \oint_{S} \vec{B} \cdot d\vec{S} = 0$ &
        $\displaystyle \vec{\nabla} \cdot \vec{B} = 0$ \\

      {\bf Ampere's law} &
        $\displaystyle \oint_{L} \vec{H} \cdot d\vec{\ell} =  \int_{S(L)} \vec{j}_{free} \cdot d\vec{S} = I_{free}$ &
        $\displaystyle \vec{\nabla} \times \vec{H} = \vec{j}_{free}$ \\
      \hline
    \end{tabular}
  \end{table}
\end{center}
}

\subsubsection*{\bf Maxwell's equations - Dynamic case (time-dependent fields) in vacuum}

{\small
\begin{center}
{
  \begin{table}[H]
    \begin{tabular}{|l|c|c|}
      % \hline
      %   \multicolumn{3}{|l|} {
      %     {\color{magenta}
      %      {\bf Dynamic case (time-dependent fields) in vacuum}
      %     }
      %   }\\
      \hline
      {\bf Gauss's law} &
        $\Phi_{E} = \displaystyle \oint_{S} \vec{E} \cdot d\vec{S} = \frac{1}{\epsilon_0} \int_{\tau(S)} \rho d\tau = \frac{Q}{\epsilon_0}$ &
        $\displaystyle \vec{\nabla} \cdot \vec{E} = \frac{\rho}{\epsilon_0}$ \\

      {\bf Faraday's law} &
        $\displaystyle \oint_{L} \vec{E} \cdot d\vec{\ell} =
           -\frac{\partial}{\partial t} \int_{S(L)} \vec{B} \cdot d\vec{S} = -\frac{\partial\Phi^{open}_{B}}{\partial t}$ &
        $\displaystyle \vec{\nabla} \times \vec{E} = -  \frac{\partial \vec{B}}{\partial t}$ \\

      {\bf Gauss's law} (magn) &
        $\Phi_{B} = \displaystyle \oint_{S} \vec{B} \cdot d\vec{S} = 0$ &
        $\displaystyle \vec{\nabla} \cdot \vec{B} = 0$ \\

      {\bf Ampere's law} &
        $\displaystyle \oint_{L} \vec{B} \cdot d\vec{\ell} =
           \mu_{0} \int_{S(L)} \Big( \vec{j} + \epsilon_0 \frac{\partial \vec{E}}{\partial t}\Big) \cdot d\vec{S}$ &
        $\displaystyle \vec{\nabla} \times \vec{B} =
           \mu_{0} \Big( \vec{j} + \epsilon_0 \frac{\partial \vec{E}}{\partial t}\Big)$ \\
      \hline
    \end{tabular}
  \end{table}
}
\end{center}
} %small


\subsubsection*{\bf Maxwell's equations - Dynamic case (time-dependent fields) in matter}

{\small
 \begin{center}
 {
  \begin{table}[H]
    \begin{tabular}{|l|c|c|}
      % \hline
      %   \multicolumn{3}{|l|} {
      %     {\color{magenta}
      %      {\bf Dynamic case (time-dependent fields) in matter}
      %     }
      %   }\\
      \hline
      {\bf Gauss's law} &
        $\Phi_{D} = \displaystyle \oint_{S} \vec{D} \cdot d\vec{S} =
          \int_{\tau(S)} \rho_{free} d\tau = Q_{free}$   &
        $\displaystyle \vec{\nabla} \cdot \vec{D} =
          \rho_{free}$ \\

      {\bf Faraday's law} &
        $\displaystyle \oint_{L} \vec{E} \cdot d\vec{\ell} =
            -\frac{\partial}{\partial t} \int_{S(L)} \vec{B} \cdot d\vec{S} = -\frac{d\Phi^{open}_{B}}{dt}$ &
        $\displaystyle \vec{\nabla} \times \vec{E} =
            - \frac{\partial \vec{B}}{\partial t}$ \\

      {\bf Gauss's law} (magn.) &
        $\displaystyle  \Phi_{B} = \oint_{S} \vec{B} \cdot d\vec{S} = 0$ &
        $\displaystyle  \vec{\nabla} \cdot \vec{B} = 0$ \\

      {\bf Ampere's law} &
        $\displaystyle \oint_{L} \vec{H} \cdot d\vec{\ell} =
           \int_{S(L)} \Big( \vec{j_{free}} + \frac{\partial \vec{D}}{\partial t}\Big) \cdot d\vec{S} = I_{free} + \frac{d\Phi^{open}_{D}}{dt}$ &
        $\displaystyle \vec{\nabla} \times \vec{H} = \vec{j}_{free} + \frac{\partial \vec{D}}{\partial t}$ \\
      \hline
    \end{tabular}
  \end{table}

 }
 \end{center}
} %small


\subsubsection*{\bf The scalar potential V - The Poisson and Laplace equations}

In electrostatics, the circuital law:
\begin{equation*}
      \vec{\nabla} \times \vec{E} = 0
\end{equation*}
allowed us to write the electric field $\vec{E}$ as the gradient of a scalar field
(recall that $\vec{\nabla} \times \Big( \vec{\nabla} \phi \Big) =0$
for {\em any} scalar function $\phi$).
That scalar field was the potential V:
\begin{equation*}
      \vec{E} = - \vec{\nabla} V
\end{equation*}
By substituting the above into Gauss's law,
we found that the scalar potential V satisfies the so-called {\em Poisson} equation:
\begin{equation*}
      \vec{\nabla}^{2} V = - \frac{\rho}{\epsilon_0}
\end{equation*}

In absence of sources ($\rho$=0), the above equation is known as the {\em Laplace} equation:
\begin{equation*}
      \vec{\nabla}^{2} V = 0
\end{equation*}


\subsubsection*{\bf The vector potential $\vec{A}$}

The relation
\begin{equation*}
      \vec{\nabla} \cdot \vec{B} = 0
\end{equation*}
allows us to express the magnetic field $\vec{B}$ as the rotation of a
vector field (recall that $\vec{\nabla} \cdot \Big( \vec{\nabla} \times \vec{A} \Big) =0$
for {\em any} vector field $\vec{A}$):

\begin{equation*}
      \vec{B} = \vec{\nabla} \times \vec{A}
\end{equation*}
We call $\vec{A}$ the {\bf vector potential}.

Substituting the above definition into Ampere's law:
\begin{equation*}
     \vec{\nabla} \times \vec{B} = \mu_{0} \vec{j} \Rightarrow
     \vec{\nabla} \times \Big( \vec{\nabla} \times \vec{A}  \Big) =  \mu_{0} \vec{j} \Rightarrow
\end{equation*}
\begin{equation*}
     \vec{\nabla} \Big( \vec{\nabla} \cdot \vec{A}  \Big) - \vec{\nabla}^2 \vec{A}   =  \mu_{0} \vec{j}
\end{equation*}

We can add to $\vec{A}$  {\em any function} $\vec{\Lambda}$
whose curl vanishes ($\vec{\nabla} \times \vec{\Lambda} = 0$):
\begin{equation*}
   \vec{A} \rightarrow \pvec{A}' = \vec{A} + \vec{\Lambda}
\end{equation*}
and the physics would remain unchanged!
We can use this freedom to eliminate the divergence of
$\vec{A}$ ($\vec{\nabla} \vec{A} = 0$).
With this condition, $\vec{A}$ satisfies the following p.d.e.:
\begin{equation*}
   \vec{\nabla}^2 \vec{A}   =  - \mu_{0} \vec{j}
\end{equation*}

This expression is very similar to our known {\em Poisson} equation.


\subsubsection*{\bf Capacitance}
Capacitance (C) denotes the ability of a body to store electric charge.
For a system of two conductors, one with charge +Q held at
potential $V_{+}$ and one with charge -Q held at potential $V_{-}$,
the capacitance is a positive quantity defined as:
\begin{equation*}
       C = \frac{Q}{V_{+} - V_{-}}
\end{equation*}

\subsubsection*{\bf Calculating the capacitance for simple systems}

\begin{itemize}
{\small
       \item Use Gauss' law to calculate the electric field $\vec{E}$
                 in terms of the charge Q stored in one of the conductors:
                $\oint_{S} \vec{E} \cdot d\vec{S} = Q/\epsilon_0$

       \item Once $\vec{E}$ is known, calculate the potential
                 difference V between the two conductors as:
                 $V := {\Delta}V = V_{+} - V_{-} = - \int_{-}^{+} \vec{E} \cdot d\vec{\ell}$

        \item From the known charge Q in the positive conductor and
                  the potential difference V between the conductors,
                  calculate $C = Q/V$
}
\end{itemize}

\subsubsection*{\bf The parallel plate capacitor}
\begin{itemize}
{\small
\item The electric field between the plates was found to be:
   \begin{equation*}
     E = \frac{\sigma}{\epsilon_0}
   \end{equation*}

\item The parallel plate capacitor has capacitance:
   \begin{equation*}
       C = \epsilon_0 \frac{A}{d}
   \end{equation*}
   It depends only on the geometrical characteristics of the capacitor.

\item Energy stored in the parallel plate capacitor:
   \begin{equation*}
       U = \frac{Q^2}{2C} \;\;\;\; or \;\;\;\;
       U = \frac{1}{2} C V^2 \;\;\;\; or \;\;\;\;
       U = \frac{1}{2} Q V
   \end{equation*}
}
\end{itemize}

\subsubsection*{\bf Mutual and self inductance / Back-EMF}

Consider two closed loops 1 and 2.
The magnetic flux $\Phi_2$ through the surface of loop 2, of the magnetic field $\vec{B}_1$ produced by a steady current $I_1$ in loop 1 is
\begin{equation*}
   \Phi_2 = M  I_1
\end{equation*}
where M is the mutual inductance of the two loops.
Similarly,
the magnetic flux $\Phi_1$ through the surface of loop 1, of the magnetic field $\vec{B}_2$ produced by a steady current $I_2$ in loop 2 is
\begin{equation*}
   \Phi_1 = M  I_2
\end{equation*}
Note: The constant M is the same (mutual inductuance)


If there is a change in the flow of current in one of the conductors,
then a voltage (EMF) is induced both
\begin{itemize}
  { \scriptsize
   \item in itself (self-inductance): \\
             $\displaystyle \mathcal{E} = - L \frac{dI}{dt}$
   \item and the neighbouring conductor (mutual inductance):\\
            $\displaystyle \mathcal{E}_{neighbouring\;loop} = - M \frac{dI}{dt}$
  }
\end{itemize}


\subsubsection*{\bf The $\vec\nabla$ ({\em nabla}) operator}
  \begin{equation*}
     \vec{\nabla} = (\frac{\partial}{\partial x}, \frac{\partial}{\partial y}, \frac{\partial}{\partial z})
  \end{equation*}
   Just as we have 3 kinds of vector multiplications (with scalar, dot product, cross product),
   we have 3 ways the nabla operator can act
    \begin{itemize}
       \item $\vec\nabla$ (a scalar field) $\rightarrow$ {\bf gradient}
       \item $\vec\nabla$ $\cdot$ (a vector field) $\rightarrow$ {\bf divergence}
       \item $\vec\nabla$ $\times$ (a vector field) $\rightarrow$ {\bf curl}
    \end{itemize}

\subsubsection*{\bf Gradient}
Let $f(x,y,z)$ be a scalar field in a 3-dimensional space. Its gradient is:
\begin{equation*}
   \vec\nabla f(x,y,z) =
     \Big(
       \frac{\partial f(x,y,z)}{\partial x},
       \frac{\partial f(x,y,z)}{\partial y},
       \frac{\partial f(x,y,z)}{\partial z}
     \Big)
\end{equation*}
Notice that $\vec\nabla f(x,y,z)$ is a vector.
It tells us how fast $f(x,y,z)$ changes as we move in space and it has the {\bf direction of the steepest ascent}.\\

\subsubsection*{\bf Divergence}
Let $\vec{F}(x,y,z)=\Big(F_{x}(x,y,z), \; F_{y}(x,y,z), \; F_{z}(x,y,z) \Big)$
be a vector field in a 3-dimensional space. Its divergence is given by:
\begin{equation*}
   \vec\nabla \cdot \vec{F}(x,y,z) =
       \frac{\partial F_{x}(x,y,z)}{\partial x} +
       \frac{\partial F_{y}(x,y,z)}{\partial y} +
       \frac{\partial F_{z}(x,y,z)}{\partial z}
\end{equation*}
Notice that the divergence of a vector field is a scalar.\\
Its value for a particular point expresses the magnitude of the vector field's source (or sink) at that point.

\subsubsection*{\bf Curl}
The curl ($\vec{\nabla} \times \vec{A}$) of a vector
field $\vec{A}$ = $\Big(A_x, A_y, A_z \Big)$ is defined as:
\begin{equation*}
     \vec{\nabla} \times \vec{A} =
       \left|
          \begin{array}{ccc}
             \hat{x} & \hat{y} & \hat{z} \\
             \frac{\partial}{\partial x}  & \frac{\partial}{\partial y} & \frac{\partial}{\partial z} \\
             A_x     & A_y     & A_z     \\
          \end{array}
       \right| =
       \Big( \frac{\partial A_z}{\partial y} - \frac{\partial A_y}{\partial z} \Big) \hat{x} -
       \Big( \frac{\partial A_z}{\partial x} - \frac{\partial A_x}{\partial z} \Big) \hat{y} +
       \Big( \frac{\partial A_y}{\partial x} - \frac{\partial A_x}{\partial y} \Big) \hat{z}
\end{equation*}
Notice that the curl of a vector field is a vector.\\
Its value for a particular point expresses the rotation of the vector field
around that point.

\subsubsection*{\bf The Laplace operator}

The Laplace operator is a 2$^{nd}$ order differential operator defined as follows:
\begin{equation*}
  \vec{\nabla}^{2} = \frac{\partial^2}{\partial x^2} + \frac{\partial^2}{\partial y^2} + \frac{\partial^2}{\partial z^2}
\end{equation*}

\subsubsection*{\bf Laplacian}

So the Laplacian of a scalar function f is:
\begin{equation*}
  \vec{\nabla}^{2}f = \frac{\partial^2 f}{\partial x^2} + \frac{\partial^2 f}{\partial y^2} + \frac{\partial^2 f}{\partial z^2}
\end{equation*}

The Laplacian {\bf is the divergence of the gradient} of the scalar function f:
\begin{equation*}
  \vec{\nabla}^{2} f =  \vec{\nabla} \cdot \Big( \vec{\nabla} f \Big)
\end{equation*}

It represents a quantity which is important in several physical processes:\\
The Laplacian $\vec{\nabla}^{2} f(\vec{r})$ of a scalar function f at a point $\vec{r}$
tells you {\bf how much $f(\vec{r})$ differs from its average over a small volume around $\vec{r}$.}

\subsubsection*{\bf Difference between total and partial derivatives}
\begin{equation*}
   \frac{df(x,y,z,...)}{dx} =
     \frac{\partial f(x,y,z,...)}{\partial x} +
     \frac{\partial f(x,y,z,...)}{\partial y} \cdot \frac{\partial y}{\partial x} +
     \frac{\partial f(x,y,z,...)}{\partial z} \cdot \frac{\partial z}{\partial x} + ...
\end{equation*}

\subsubsection*{\bf Gauss's theorem (Divergence theorem)}
Let $\tau$ be a volume of space whose boundary is the closed surface S,
and let $\vec{F}$ be a vector field which is defined anywhere in $\tau$
and it is continuous / differentiable anywhere in $\tau$.\\
The integral of the flux of the field $\vec{F}$ over the closed surface S is
equal to the integral of the divergence of $\vec{F}$ in the volume $\tau$\\
\begin{equation*}
  \oint_{S} \vec{F} \cdot d\vec{S} = \int_{\tau(S)} \vec{\nabla} \cdot \vec{F} d\tau
\end{equation*}

\subsubsection*{\bf Stokes's theorem}
Stokes' theorem {\bf relates
the circulation of a vector field $\vec{F}$ around a closed line C
with the flux of the curl of the vector field $\vec{F}$ through the open surface S} defined by the closed line C.\\
\begin{equation*}
  \oint_{L} \vec{F} \cdot d\vec{\ell} = \int_{S(L)} (\vec{\nabla} \times \vec{F}) \cdot d\vec{S}
\end{equation*}

\end{document}
