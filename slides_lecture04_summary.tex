
\renewcommand{\summarizedlecture}{4 }

%
%
%

\begin{frame}{Lecture \summarizedlecture - \lecturesummarytitle}

\begin{itemize}

  \item With regards to electrical properties, there are 2 types of materials
   \begin{itemize}
     \item materials that conduct electricity: {\bf conductors}
     \item materials that do not conduct electricity: {\bf insulators (dielectrics)}
   \end{itemize}

   \vspace{0.2cm}

   \item A conductor is an object or type of material
         which {\bf contains electric charges that are relatively free to move}
  \begin{itemize}
     \item A {\em \bf perfect conductor} has an {\bf unlimited supply of free charges}.
     \item There are no perfect conductors, but many substances come close!
      \begin{itemize}
           \item e.g. the free charge density in copper is 1.8 $\times$ 10$^{10}$ C/$m^3$.
      \end{itemize}
   \end{itemize}

   \vspace{0.2cm}

   \item If we place a {\bf conductor within an external electric field}:
     \begin{itemize}
       \item The electric field vanishes everywhere inside a conductor.
       \item The potential is constant inside a conductor.
       \item Charge accumulates in the surface.
       \item The field on the surface of a conductor has no tangential component.
     \end{itemize}

\end{itemize}

\end{frame}

%
%
%

\begin{frame}{Lecture \summarizedlecture - \lecturesummarytitle (cont'd)}

\begin{itemize}

   \item Capacitance (C) denotes the ability of a body to store electric charge.
             For a system of two conductors, one with charge +Q held at
             potential $V_{+}$ and one with charge -Q held at potential $V_{-}$,
             the capacitance is a positive
             quantity defined as:
      \begin{equation*}
          C = \frac{Q}{V_{+} - V_{-}}
      \end{equation*}
      Its SI unit is the {\bf Farad (F)} defined as one Coulomb per Volt.

   \vspace{0.1cm}

   \item Calculating the capacitance for simple systems:

        \begin{itemize}
         {\small
              \item Use Gauss' law to calculate the electric field $\vec{E}$
                        in terms of the charge Q stored in one of the conductors:
                       $\oint_{S} \vec{E} d\vec{S} = Q/\epsilon_0$

              \item Once $\vec{E}$ is known, calculate the potential
                        difference V between the two conductors as:
                        $V := {\Delta}V = V_{+} - V_{-} = - \int_{-}^{+} \vec{E} d\vec{\ell}$

               \item From the known charge Q in the positive conductor and
                         the potential difference V between the conductors,
                         calculate $C = Q/V$
        }
        \end{itemize}

\end{itemize}

\end{frame}


%
%
%

\begin{frame}{Lecture \summarizedlecture - \lecturesummarytitle (cont'd)}

\begin{itemize}

   \item We studied a simple system: The {\bf parallel plate capacitor}

   \item The electric field between the plates was found to be:
      \begin{equation*}
        E = \frac{\sigma}{\epsilon_0}
      \end{equation*}

   \item The parallel plate capacitor has capacitance:
      \begin{equation*}
          C = \epsilon_0 \frac{A}{d}
      \end{equation*}
      It depends only on the geometrical characteristics of the capacitor.

   \item Energy stored in the parallel plate capacitor:
      \begin{equation*}
          U = \frac{Q^2}{2C} \;\;\;\; or \;\;\;\;
          U = \frac{1}{2} C V^2 \;\;\;\; or \;\;\;\;
          U = \frac{1}{2} Q V
      \end{equation*}
      and confirmed that:
      \begin{equation*}
          U = \frac{\epsilon_0}{2} \int_{all\;space} |\vec{E}(\vec{r})|^2  d\tau
      \end{equation*}

\end{itemize}

\end{frame}

%
%
%

\begin{frame}{Lecture \summarizedlecture - \lecturesummarytitle (cont'd)}

\begin{itemize}

   \item  {\bf Electric dipole}:
          Point charges +q and -q at a {\em small distance} d.\\

          \vspace{0.1cm}

   \item  An electric dipole its described by its {\bf electric dipole moment} $\vec{p} = q \vec{d}$\\
          \begin{itemize}
                 \item A vector pointing from the negative to the positive charge
           \end{itemize}

          \vspace{0.1cm}

   \item  An electric dipole creates a potential field V given by
          \begin{equation*}
            V \approx \frac{1}{4\pi\epsilon_0} \frac{\vec{p} \cdot \hat{r}}{r^2}
          \end{equation*}

   \item  An electric field $\vec{E}$ exerts to an electric dipole with moment $\vec{p}$
          a torque $\vec{T} = \vec{p} \times \vec{E}$

          \vspace{0.1cm}

   \item  Electric fields induce dipole moments in the direction of the field (or align
              towards the direction of the field polar molecules with permanent dipole moments)
              and generate macroscopic polarisation.

          \vspace{0.1cm}

   \item  The {\bf polarisation} $\vec{P}$ of a material is defined as the
          {\bf amount of electric dipole moment per unit volume}.\\

\end{itemize}

\end{frame}


%
%
%

\begin{frame}{Lecture \summarizedlecture - \lecturesummarytitle (cont'd)}

\begin{itemize}

   \item  The polarisation induces surface and volume polarisation charges.
          The corresponding densities are
          $\sigma_P = \vec{P} \cdot \hat{n}$ and $\rho_P = - \vec{\nabla} \vec{P}$

          \vspace{0.1cm}

    \item In the presence of dielectrics,  Gauss's law need to be generalised
          to include both free charges we also have induced polarisation charges:
          \begin{equation*}
             \vec{\nabla} \Big( \epsilon_0 \vec{E} + \vec{P} \Big) = \rho_f
              \;\;\;\; and \;\;\;\;
              \oint  \Big( \epsilon_0 \vec{E} + \vec{P} \Big) d\vec{S} = Q_f
          \end{equation*}
          where $\rho_f$ is the free charge density and $Q_f$ the amount of free charge.

          \vspace{0.1cm}

    \item The vector
           $\vec{D} = \epsilon_0 \vec{E} + \vec{P}$ is the electric displacement vector
          \begin{itemize}
             \item In SI, the electric displacement unit is $C/m^2$.
          \end{itemize}

          \vspace{0.1cm}

    \item For {\bf linear dielectrics},
              $\vec{P} = \chi_{e} \epsilon_0 \vec{E}$ and, therefore,
              $\vec{D} = \epsilon_r \epsilon_0 \vec{E}$,
              where
              $\chi_{e}$ is the electric susceptibility of the material and
              $\epsilon_r = 1 + \chi_e$ is the relative permittivity (dielectric constant).

 \item A dielectric with relative permittivity $\epsilon_r$ inserted
         between the plates of a parallel plate capacitor, increases
         its capacitance by a factor of $\epsilon_r$.

\end{itemize}

\end{frame}

%
%
%

\begin{frame}{Maxwell's equation we know so far}

In vacuum (static case):

\begin{center}
 {
  \begin{table}[H]
    \begin{tabular}{|l|c|c|}
      \hline
          & {\it Integral form} & {\it Differential form} \\
      \hline
      {\bf Gauss's law} &
        $\oint \vec{E} d\vec{S} = Q_{enclosed} / \epsilon_0$ &
        $\vec{\nabla} \cdot \vec{E} = \rho / \epsilon_0$ \\

      {\bf Circuital law} &
        $\oint \vec{E} d\vec{\ell} = 0$ &
        $\vec{\nabla} \times \vec{E} = 0$ \\
      \hline
    \end{tabular}
  \end{table}
 }
\end{center}

In the presence of materials (static case):

\begin{center}
 {
  \begin{table}[H]
    \begin{tabular}{|l|c|c|}
      \hline
          & {\it Integral form} & {\it Differential form} \\
      \hline
      {\bf Gauss's law} &
        $\oint \vec{D} d\vec{S} = Q_{enclosed; free}$ &
        $\vec{\nabla} \cdot \vec{D} = \rho_{free}$ \\

      {\bf Circuital law} &
        $\oint \vec{E} d\vec{\ell} = 0$ &
        $\vec{\nabla} \times \vec{E} = 0$ \\
      \hline
    \end{tabular}
  \end{table}
 }
\end{center}

\end{frame}
