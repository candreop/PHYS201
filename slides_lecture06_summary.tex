
\renewcommand{\summarizedlecture}{6 }


\begin{frame}{Lecture \summarizedlecture - \lecturesummarytitle}

\begin{itemize}

\item
  At distance $\rho$, the force between two parallel straight
  conductors of length L carrying current $I_1$ and $I_2$,
  respectively, is:
  \begin{equation*}
    F = \frac{\mu_0 I_{1} I_{2} L}{2\pi \rho}
  \end{equation*}
  The force is attractive if both currents flow in the same direction, and
  repulsive if the two currents flow in opposite directions.

\item
  One Ampere is the amount of current that,
  if maintained between those conductors produces a force of 2
  $\times$ $10^{-7}$ N per metre of length.

\item
  The magnetic dipole moment $\vec{m}$ of a current loop is defined as
  \begin{equation*}
      \vec{m} = I \vec{S}
   \end{equation*}

\item
  The magnetic moment $\vec{T}$ of a magnet is a quantity that
  determines the torque it will experience in an external magnetic field.
  \begin{equation*}
      \vec{T} = \vec{m} \times \vec{B}
  \end{equation*}

\end{itemize}

\end{frame}

%
%
%

\begin{frame}{Lecture \summarizedlecture - \lecturesummarytitle (cont'd)}

\begin{itemize}

\item
    {\bf Ampere's law} in integral form:
     \begin{equation*}
        \oint_{L} \vec{B} \cdot d\vec{\ell} = \mu_0 I_{encl} = \mu_0 \int_{S} \vec{j} \cdot d\vec{S}
     \end{equation*}
     The line integral of the magnetic field $\vec{B}$ along a closed path L is proportional to the
     current passing though any surface S defined by L.

\vspace{0.2cm}

\item
    {\bf Ampere's law} in differential form:
    \begin{equation*}
       \vec{\nabla} \times \vec{B} = \mu_0 \vec{j}
    \end{equation*}
    The curl $\vec{B}$ is proportional to the local current density $\vec{j}$.

\vspace{0.2cm}

\item
    The magnetic field inside a toroidal coil with n windings, each carrying current I, is:
     \begin{equation*}
          |\vec{B}| = \frac{\mu_0 n I}{2\pi r}
     \end{equation*}
     where r is the distance from the centre of the coil.

\end{itemize}

\end{frame}

%
%
%

\begin{frame}{Lecture \summarizedlecture - \lecturesummarytitle (cont'd)}

\begin{itemize}

\item
      The relation:
      \begin{equation*}
              \vec{\nabla} \cdot \vec{B} = 0
      \end{equation*}
      allows us to express $\vec{B}$ as the curl of a vector field  $\vec{A}$:
      \begin{equation*}
             \vec{B} = \vec{\nabla} \times \vec{A}
      \end{equation*}
      We call $\vec{A}$ the {\bf vector potential}.

\item
     There is some freedom in determining $\vec{A}$:
    \begin{itemize}
         \item Adding to it a function whose curl is zero leaves the physics unchanged.
         \item We use this freedom to eliminate the divergence of $\vec{A}$ ($\vec{\nabla} \cdot \vec{A} = 0$).
    \end{itemize}

\item
     The vector potential $\vec{A}$ satisfies the following equation:
     \begin{equation*}
         \vec{\nabla}^2 \vec{A}   =  - \mu_{0} \vec{j}
     \end{equation*}
     (each component of $\vec{A}$ satisfies a Poisson equation).

\end{itemize}

\end{frame}
