
\renewcommand{\summarizedlecture}{12 }

%
%
%

\begin{frame}{Lecture \summarizedlecture - \lecturesummarytitle}

{\small

In an RLC circuit, the phase $\phi$ by which the voltage leads the current is given by:
\begin{equation*}
     tan\phi = \frac{V_L - V_C}{V_R} \Rightarrow {\color{magenta}  tan\phi = \frac{X_L - X_C}{R} }
\end{equation*}

\begin{itemize}
  \item  {\color{magenta}  $X_L > X_C$}: The circuit is {\bf more inductive than capacitive} (voltage leads).
  \item  {\color{magenta}  $X_L < X_C$}: The circuit is {\bf more capacitive then inductive} (current leads).
  \item  {\color{magenta}  $X_L = X_C$}: the circuit is {\bf in resonance}
      \begin{itemize}
       {\scriptsize
          \item min. impedance,  max. current, and current in phase with voltage in ($\phi=0$)
       }
      \end{itemize}
\end{itemize}

The resonance happens when the {\bf driving angular frequency matches the natural angular frequency}:
\begin{equation*}
  X_L = X_C \Rightarrow \omega = \frac{1}{\sqrt{LC}}
\end{equation*}

}
\end{frame}
