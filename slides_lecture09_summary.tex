
\renewcommand{\summarizedlecture}{9 }

%
%
%

\begin{frame}{Lecture \summarizedlecture - \lecturesummarytitle}

\begin{itemize}

\item
We looked at Maxwell's equation in vacuum in absence of sources, and we
saw that {\bf the electric and magnetic fields satisfy a wave equation}:
\begin{equation*}
  \vec{\nabla}^{2} \vec{E} = \mu_0 \epsilon_0 \frac{\partial^{2} \vec{E}}{\partial t^{2}}
  \;\;\;\; and \;\;\;\;
  \vec{\nabla}^{2} \vec{B} = \mu_0 \epsilon_0 \frac{\partial^{2} \vec{B}}{\partial t^{2}}
\end{equation*}

\item
The velocity of an electromagnetic wave is given by:
\begin{equation*}
    \frac{1}{\sqrt{\mu_0 \epsilon_0}} = {\bf 299,792,458 \; m/s}  \;\;\;(the \; speed \; of \; light)
\end{equation*}

\item
This led Maxwell to realize that {\bf light is an electromagnetic wave}.

\end{itemize}

\end{frame}

%
%
%

\begin{frame}{Lecture \summarizedlecture - \lecturesummarytitle (cont'd)}

\begin{itemize}

\item
An electromagnetic wave is a {\bf most peculiar wave}!

   \begin{itemize}
   {\small
     \item An electromagnetic wave  {\bf can not die off}!
             \begin{itemize}
             {\scriptsize
               \item A time-varying electric field generates a time-varying magnetic field.
               \item A time-varying magnetic field generates a time-varying electric  field.
             }
             \end{itemize}
             {\bf The change of one field feeds the other!}

     \vspace{0.1cm}

     \item An electric wave can't exist without a magnetic one (and vice versa)\\
               They are part of the same phenomenon: {\bf electromagnetic waves}.

     \vspace{0.1cm}

     \item The electric and magnetic fields are perpendicular to each
               other an perpendicular to the direction of wave propagation.

     \vspace{0.1cm}

     \item Electromagnetic waves {\bf don't need a medium to propagate}!
   }
   \end{itemize}

\item
The {\bf Poynting vector} $\vec{N}$ as:
\begin{equation*}
  \vec{N} = \frac{1}{\mu_0} \Big( \vec{E} \times \vec{B} \Big)
\end{equation*}
It represents the {\bf energy flux density (rate energy transfer per unit area)} and it has units of $W(att)/m^2$.\\

\end{itemize}

\end{frame}


%
%
%

\begin{frame}{Lecture \summarizedlecture - \lecturesummarytitle (cont'd)}

\begin{itemize}

\item
E/M waves {\bf carry momentum}:
Can {\bf exert pressure} on an object when shining on it!\\
\item
If an object is illuminated by radiation for a time interval ${\Delta}t$,
during which it absorbs energy ${\Delta}U$ = IA${\Delta}t$,
where I is the intensity I (power per area, or energy per time
per area) of the radiation, and A is the area of the object, then:
\begin{itemize}
\item
The momentum change ${\Delta}p$ of the object is given by:
\begin{equation*}
    {\Delta}p = \frac{{\Delta}U}{c}
    \;\;\; {\color{red} \small (total \; absorption)}
    \;\;\;
    {\Delta}p = \frac{2{\Delta}U}{c}
    \;\;\; {\color{red} \small (total \; reflection)}
\end{equation*}
\item
The radiation pressure $P_r$ exterted on the object is given by:
\begin{equation*}
  P_r = \frac{I}{c}
  \;\;\; {\color{red} \small (total \; absorption)}
  \;\;\;
  P_r = \frac{2I}{c}
  \;\;\; {\color{red} \small (total \; reflection)}
\end{equation*}
\end{itemize}

\end{itemize}

\end{frame}
