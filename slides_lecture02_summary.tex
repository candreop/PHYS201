\renewcommand{\summarizedlecture}{2 }


%
%
%

\begin{frame}{Lecture \summarizedlecture - \lecturesummarytitle}

\begin{itemize}
{\small

\item {\bf Electric flux}
  \begin{itemize}
  {\small
     \item The electric flux $\Phi_E$ is the number of field lines of the electric field $\vec{E}$
           flowing through a surface S
        \begin{equation*}
          \Phi_E = \int_{S} \vec{E} \cdot d\vec{S}
        \end{equation*}
  }
  \end{itemize}

\item {\bf Gauss' law}
  \begin{itemize}
  {\small
     \item Our first Maxwell equation!
     \item In integral form (useful if a symmetry can be exploited to simplify the integral evaluation):
           Relates the flux through a closed surface with the net charge contained in it
           \begin{equation*}
              \oint_{S} \vec{E} \cdot d\vec{S} = \frac{Q_{enc}}{\epsilon_0}
           \end{equation*}
     \item In differential form:
           Relates the divergence of the electric field with the local charge density
           \begin{equation*}
              \vec{\nabla} \cdot \vec{E} = \frac{\rho}{\epsilon_0}
           \end{equation*}
  }
  \end{itemize}

}
\end{itemize}

\end{frame}
