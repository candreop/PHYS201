
\renewcommand{\summarizedlecture}{10 }

%
%
%

\begin{frame}{Lecture \summarizedlecture - \lecturesummarytitle}

{\small

We studied the most general case of Maxwell's equations:\\

\setlength{\extrarowheight}{12pt}
\setlength{\arraycolsep}{5pt}

 \begin{center}
 {

  \begin{table}[H]
    \begin{tabular}{|l|c|c|}
      \hline
        \multicolumn{3}{|l|} {
          {\color{magenta}
           {\bf Dynamic case in matter}
          }
        }\\
      \hline
      {\bf Gauss's law} &
        $\displaystyle \oint \vec{D} \cdot d\vec{S} =
          \int \rho_f d\tau = Q_f$   &
        $\displaystyle \vec{\nabla} \cdot \vec{D} =
          \rho_f$ \\

      {\bf Faraday's law} &
        $\displaystyle \oint \vec{E} \cdot d\vec{\ell} =
            -\frac{\partial}{\partial t} \int \vec{B} \cdot d\vec{S} \Rightarrow$ &
        $\displaystyle \vec{\nabla} \times \vec{E} =
            - \frac{\partial \vec{B}}{\partial t}$ \\
      &
        $\displaystyle \oint \vec{E} \cdot d\vec{\ell} =
            -\frac{d\Phi_B}{dt}$ & \\

      {\bf Gauss's law} (magn.) &
        $\displaystyle  \oint \vec{B} \cdot d\vec{S} = 0$ &
        $\displaystyle  \vec{\nabla} \cdot \vec{B} = 0$ \\

      {\bf Ampere's law} &
        $\displaystyle \oint \vec{H} \cdot d\vec{\ell} =
           \int_{S} \Big( \vec{j} + \frac{\partial \vec{D}}{\partial t}\Big) \cdot d\vec{S} \Rightarrow$ &
        $\displaystyle \vec{\nabla} \times \vec{H} = \vec{j}_f + \frac{\partial \vec{D}}{\partial t}$ \\
      &
        $\displaystyle \oint \vec{H} \cdot d\vec{\ell} = I_f + \frac{d\Phi_D}{dt}$ & \\
      \hline
    \end{tabular}
  \end{table}

 }
 \end{center}
}

\end{frame}

%
%
%

\begin{frame}{Lecture \summarizedlecture - \lecturesummarytitle (cont'd)}

\begin{itemize}

\item As we did in vacuum, we studied Maxwell's equations in matter (for time-dependent fields)
          and in the absence of sources and we saw that they give rise to EM waves:
          \begin{equation*}
             \vec{\nabla}^{2} \vec{E} = \mu \epsilon \frac{\partial^{2} \vec{E}}{\partial t^{2}} \;\;\;\; and \;\;\;\;
             \vec{\nabla}^{2} \vec{B} = \mu \epsilon \frac{\partial^{2} \vec{B}}{\partial t^{2}}
         \end{equation*}

\item Speed of EM waves in matter: $\displaystyle u = \frac{1}{\sqrt{\epsilon \mu}}$
         \begin{itemize}
         {\small
               \item EM waves in matter propagate slower than EM waves in vacuum
               \item $\displaystyle u = \frac{c}{n}$ where
                         $\displaystyle n = \frac{\sqrt{\epsilon \mu}}{\sqrt{\epsilon_0 \mu_0}}$
                         is the index of refraction of the material.
         }
         \end{itemize}

\item We introduced a complex representation of EM waves starting from de Moivre's theorem
          and embedding the known EM wave properties (EM waves are always {\bf transverse} and
          {\bf mutually perpendicular}):
          \begin{equation*}
                \vec{E}(\vec{r},t) = E_0  e^{i (\vec{k} \vec{r} -\omega t)} \hat{n}
                 \;\;\; and \;\;\;
                \vec{B}(\vec{r},t) =
                  \frac{E_0}{c}  e^{i ( \vec{k} \vec{r} -\omega t)} \Big( \hat{k} \times \hat{n} \Big) =
                  \frac{1}{c}  \Big( \hat{k} \times \vec{E} \Big)
          \end{equation*}

\end{itemize}

\end{frame}

%
%
%

\begin{frame}{Lecture \summarizedlecture - \lecturesummarytitle (cont'd)}

\begin{itemize}

\item We also studied EM wave polarization and practical applications.

\item
We can transform unpolarized visible light into polarized light by passing
it through a {\bf polarizing sheet}.\\

\item
If $I_0$ is the intensity of the unpolarized light,
the intensity $I$ of the transmitted light is:
\begin{equation*}
  I = \frac{1}{2}I_0
\end{equation*}

\item
If the light reaching the filter is already polarized,
the intensity $I$ of the transmitted light is:
\begin{equation*}
  I = I_0 cos^2\theta
\end{equation*}
where $\theta$ is the angle between the electric field $\vec{E}$
and the polarizing direction of the sheet.
\end{itemize}

\end{frame}

%
%
%

\begin{frame}{Lecture \summarizedlecture - \lecturesummarytitle (cont'd)}

\begin{itemize}

\item Finally, we studied the electrodynamic boundary conditions:\\
          \vspace{0.2cm}
         For the electric field:
         \begin{equation*}
               \epsilon_1 E_1^{\perp} = \epsilon_2 E_2^{\perp}  \;\;\;\; and \;\;\;\;
                E_1^{\parallel} = E_2^{\parallel}
         \end{equation*}
         For the magnetic field:
         \begin{equation*}
               B_1^{\perp} = B_2^{\perp} \;\;\;\; and \;\;\;\;
               \frac{1}{\mu_1} B_1^{\parallel} = \frac{1}{\mu_2} B_2^{\parallel}
         \end{equation*}

\item We used the above conditions to study what happens when
          an EM wave crosses the {\bf boundary between two transparent media}\\
          \vspace{0.2cm}
          Two cases:
          \begin{itemize}
               \item Normal incidence
               \item Oblique incidence (general case / home study)
          \end{itemize}
      We reproduced the laws of geometric optics!
\end{itemize}

\end{frame}


% %
% %
% %
%
% \begin{frame}{Lecture \summarizedlecture - \lecturesummarytitle (cont'd)}
%
% \begin{itemize}
% {\small
% \item Maxwell's equations in most general form (time-dependent fields in matter)
%     \begin{itemize}
%      {\scriptsize
%           \item You should know both the integral and differential forms.
%           \item You should know all variations of these equations and be able to derive
%                     one from another (dynamic $\rightarrow$ static, matter $\rightarrow$ vacuum)
%           \item No marks given for writing the wrong set of equations, even if they are all
%                     similar and interconnected.\\
%     }
%     \end{itemize}
%
% \item You should be able to show that Maxwell's equations (in vacuum or in matter) in
%           absence of sources describe EM waves
%
% \item You should know how the wave equation and the EM waves are different in vacuum and in transparent media.
%
% \item You should know how to use the complex representation of waves.
%
% \item You should know (and be able to derive) the electrodynamic boundary conditions,
%           and you should be able to use them to study reflection and transmission at normal and oblique incidence.
%
% \item You should remember the fundamental laws of geometrical optics and what is Brewster's angle.
% }
% \end{itemize}
%
% \end{frame}
