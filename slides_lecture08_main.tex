\renewcommand{\prevlecture}{7 }
\renewcommand{\thislecture}{8 }
\renewcommand{\nextlecture}{9 }

%
% Cover page
%

\title[PHYS 201 / Lecture \thislecture]
{
  PHYS 201 / Lecture \thislecture\\
  {\it
       Generalizing Maxwell's eqs for time-dependent fields: \\
      Faraday's law and Maxwell's correction to Ampere's law
  }\\
}

\author[C.Andreopoulos] {
  Professor Costas Andreopoulos\inst{1,2}, {\it FHEA}
}
\institute[Liverpool/STFC-RAL] {
   \inst{1} University of Liverpool, Department of Physics\\
   \vspace{0.1cm}
   \inst{2} U.K. Research \& Innovation (UKRI), Science \& Technology Facilities Council,\\
            Rutherford Appleton Laboratory, Particle Physics Department\\
   \vspace{0.5cm}
   {\it {\color{magenta} Lectures delivered at the University of Liverpool, 2021-22}}\\
   \vspace{0.2cm}
}
\date{\today}

\titlegraphic{
  \includegraphics[height=25px]{./images/logo/liverpool.png}
  \hspace{3px}
  \includegraphics[height=30px]{./images/logo/ral.png}
}


\begin{frame}[plain]
  \titlepage
\end{frame}

% ------------------------------------------------------------------------------
% ------------------------------------------------------------------------------

%
% Revision of previous lecture
%

\renewcommand{\lecturesummarytitle}{Revision }

\renewcommand{\summarizedlecture}{7 }

%
%
%

\begin{frame}{Lecture \summarizedlecture revision}

In the last lecture:\\

\begin{itemize}

   \item We completed the study of {\bf Maxwell's eqs. in materials (for static fields)}
             and emphasized the analogies between electrostatics and magnetostatics.

   \vspace{0.2cm}

   \item We discussed the magnetic properties of materials
             ({\bf diamagnetism}, {\bf paramagnetism} and {\bf ferromagnetism})
             and developed arguments to understand the physical origins.

   \vspace{0.2cm}

   \item We found out how to {\bf extend Maxwell's eqs. in vacuum}
             in the case of {\bf time-dependent fields}.

\end{itemize}

\end{frame}

%
%
%

\begin{frame}{Lecture \summarizedlecture revision (Magnetic moment / magnetization)}

\begin{columns}
  \begin{column}{0.25\textwidth}
    \begin{center}
      \includegraphics[width=0.95\textwidth]{./images/schematics/magnetic_dipole_moment_00.jpg}\\
    \end{center}
  \end{column}
  \begin{column}{0.70\textwidth}
    We defined the {\bf magnetic dipole moment} $\vec{m}$ as:
   \begin{equation*}
     \vec{m} = I \vec{S}
   \end{equation*}
  \end{column}
\end{columns}

\vspace{0.2cm}

An external magnetic field $\vec{B}$ exerts a torque $\vec{T}$ on a magnetic dipole $\vec{m}$ which is given by:
\begin{equation*}
  \vec{T} = \vec{m} \times \vec{B}
\end{equation*}

This will tend to {\bf align} the previously randomised {\bf magnetic moments}
and {\bf create magnetisation at a macroscopic level}.\\
\vspace{0.2cm}

We define {\bf magnetisation} $\vec{M}$ as the amount of {\bf magnetic dipole moment per unit volume}.

\end{frame}

%
%
%

\begin{frame}{Lecture \summarizedlecture revision (Magnetization-induced currents)}

The {\bf magnetisation induces surface and volume currents}.\\
\vspace{0.3cm}
We can easily be convinced, although we will not show it mathematically,
that the {\bf density of the surface current} is:
\begin{equation*}
  j_{m}^{surf} = \vec{M} \times \hat{n}
\end{equation*}

\vspace{0.1cm}

whereas the {\bf density of the volume current} is:
\begin{equation*}
  j_{m}^{vol} = \vec{\nabla} \times \vec{M}
\end{equation*}

\begin{center}
  \includegraphics[width=0.98\textwidth]{./images/schematics/magnetization_currents_01.png}\\
\end{center}

\end{frame}

%
%
%

\begin{frame}{Lecture \summarizedlecture revision (Ampere's law in materials)}

We started from Ampere's law in vacuum:
\begin{equation*}
  \vec{\nabla} \times \vec{B} = \mu_0 \vec{j}
\end{equation*}
By writing the total current density $\vec{j}$ as the vector sum of the free ($\vec{j}_{f}$)
and magnetization ($\vec{j}_{m}$) current densities,
and expressing $\vec{j}_{m}$ in terms of the
magnetization field $\vec{M}$ ($\displaystyle \vec{j}_{m} = \vec{\nabla} \times \vec{M}$),
we finally wrote Ampere's law as:
\begin{equation*}
  \vec{\nabla} \times \Big( \frac{\vec{B}}{\mu_0} - \vec{M} \Big) = \vec{j}_{f}
\end{equation*}

We defined the {\bf magnetic field strength} or {\bf magnetic field intensity} $\vec{H}$ as:
\begin{equation*}
  \vec{H} = \frac{\vec{B}}{\mu_0} - \vec{M}
\end{equation*}
In SI, the quantity $\displaystyle \vec{H} = \frac{\vec{B}}{\mu_0} - \vec{M}$ has {\bf units of A/m}.\\

\end{frame}

%
%
%

\begin{frame}{Lecture \summarizedlecture revision (Linear materials)}

Ampere's law in materials is:
$\displaystyle \vec{\nabla} \times \Big( \frac{\vec{B}}{\mu_0} - \vec{M} \Big) = \mu_0 \vec{j}_{f}$\\
\vspace{0.1cm}

If the analogy with electrostatics was exact,
we would write $\vec{M}$ in terms of $\vec{B}$.
However, this is where the analogy breaks.
Instead we typically write:
\begin{equation*}
{\color{magenta}
  \vec{M} = \chi_{m} \vec{H}
}
\end{equation*}
where  $\chi_m$ is the {\bf magnetic susceptibility}.
For {\bf linear materials}, $\chi_m$ is a constant independent of the value of $\vec{H}$.
Expressing $\vec{B}$ in terms of $\vec{H}$:
\begin{equation*}
  \vec{H} = \frac{\vec{B}}{\mu_0} - \vec{M} \Rightarrow
  \vec{B} = \mu_0 \Big( \vec{H} + \vec{M} \Big) \xRightarrow{\vec{M} = \chi_{m} \vec{H}}
  \vec{B} = \Big(1 + \chi_{m} \Big) \mu_0  \vec{H} \Rightarrow
\end{equation*}
\begin{equation*}
  \vec{B} = \mu_r \mu_0 \vec{H} \Rightarrow
  {\color{magenta}
    \vec{B} = \mu \vec{H}
  }
\end{equation*}
where $\mu_r = 1+\chi_{\mu}$
is the {\bf relative permeability} (dimensionless) and
$\mu = \mu_r  \mu_0$ is the {\bf permeability} of the material
(SI unit: $V \cdot s \cdot  A^{-1} \cdot m^{-1}$).\\

\end{frame}

%
%
%

\begin{frame}{Lecture \summarizedlecture revision (Correspondence between quantities)}

{
\setlength{\extrarowheight}{8pt}
\setlength{\arraycolsep}{5pt}

\begin{center}
  \begin{table}[H]
    \begin{tabular}{c|c||c|c}
      \hline
      \multicolumn{2}{c||}{\bf Electrostatics} &
      \multicolumn{2}{c}  {\bf Magnetostatics} \\
      \hline
         {\scriptsize electric dipole moment} &
         $\vec{p} = q \vec{d}$ &
         $\vec{m} = I \vec{S}$ &
         {\scriptsize magnetic dipole moment} \\
      \hline
         {\scriptsize torque within $\vec{E}$ field} &
         $\vec{T} = \vec{p} \times \vec{E}$ &
         $\vec{T} = \vec{m} \times \vec{B}$ &
         {\scriptsize torque within a $\vec{B}$ field} \\
      \hline
         {\scriptsize polarization} &
         $\vec{P}  = \frac{(e.d.m)}{volume}$ &
         $\vec{M} = \frac{(m.d.m)}{volume}$ &
         {\scriptsize magnetization} \\
      \hline
         {\scriptsize surface charge density} &
         $\sigma_{P} = \vec{P} \cdot \hat{n}$ &
         $j_{m}^{surf} = \vec{M} \times \hat{n}$ &
         {\scriptsize surface current density} \\
      \hline
         {\scriptsize volume charge density} &
         $\rho_{P} = - \vec{\nabla} \cdot \vec{P}$ &
         $j_{m}^{vol} = \vec{\nabla} \times \vec{M}$ &
         {\scriptsize volume current density} \\
      \hline
         {\scriptsize electric displacement} &
         $\vec{D} = \epsilon_0 \vec{E} + \vec{P}$ &
         $\vec{H} = \frac{\vec{B}}{\mu_0} - \vec{M}$ &
         {\scriptsize magnetizing field} \\
      \hline
         \multirow{2}{*}{\scriptsize Gauss' law in materials} &
         $\vec{\nabla} \cdot \vec{D} = \rho_{f}$ &
         $\vec{\nabla} \times \vec{H} = \vec{j}_{f}$ &
         \multirow{2}{*}{\scriptsize Ampere's law in materials} \\
      \hhline{~--~}
         &
         $\oint_{S} \vec{D} \cdot d\vec{S} = Q_{f}$ &
         $\oint_{L} \vec{H} \cdot d\vec{\ell} = I_{f}$ &
         \\
      \hline
    \end{tabular}
  \end{table}
\end{center}
}

\end{frame}

%
%
%

\begin{frame}{Lecture \summarizedlecture revision (Magnetic properties of materials)}

\begin{itemize}

\item We also discussed the {\bf magnetic properties of materials}
          and developed (classical) arguments to understand the {\bf physical origins}.

\item The material which has the {\bf most striking and well known
          magnetic properties is iron (Fe).}

     \begin{itemize}
            \item Nickel (Ni), Cobalt (Co), Gadolinium (Gd) and Dysprosium (Dy) behave similarly.
                      We call these materials {\bf \color{magenta} ferromagnets}.
            \item Not only these materials can have a significant magnetisation when inside an
                      external magnetic field, they also {\bf retain their magnetisation in the absence
                      of an external magnetic field.}
     \end{itemize}

\item But {\bf other substances get magnetised too} (water, wood, frogs,...)

     \begin{itemize}
             \item The magnetic effects for these materials are {\bf very very weak!}
             \item Moreover, water, wood, and frogs {\bf do not remain magnetized} once the external
                       magnetic field is removed.
             \item In a presense of an external magnetic field these substances can be magnetised
                       either in the direction of the field ({\color{magenta} {\bf paramagnetic}} substances),
                       or opposite to it ({\color{magenta} {\bf diamagnetic}} substances).
     \end{itemize}

\end{itemize}

\end{frame}

%
%
%

\begin{frame}{Lecture \summarizedlecture revision (Maxwell's eqs. for the static case)}

{\small

\begin{center}
{
  \begin{table}[H]
    \begin{tabular}{|l|c|c|}
      \hline
        \multicolumn{3}{|l|} {
          {\color{magenta}
           {\bf Static case in vacuum}
          }
        }\\
      \hline
      {\bf Gauss's law} &
        $\displaystyle \oint \vec{E} \cdot d\vec{S} = \frac{1}{\epsilon_0} \int \rho d\tau$ &
        $\displaystyle \vec{\nabla} \cdot \vec{E} = \frac{\rho}{\epsilon_0}$ \\

      {\bf Circuital law} &
        $\displaystyle \oint \vec{E} \cdot d\vec{\ell} = 0$ &
        $\displaystyle \vec{\nabla} \times \vec{E} = 0$ \\

      {\bf Gauss's law} (magn.) &
        $\displaystyle \oint \vec{B} \cdot d\vec{S} = 0$ &
        $\displaystyle \vec{\nabla} \cdot \vec{B} = 0$ \\

      {\bf Ampere's law} &
        $\displaystyle \oint \vec{B} \cdot d\vec{\ell} = \mu_{0} \int \vec{j} \cdot d\vec{S}$ &
        $\displaystyle \vec{\nabla} \times \vec{B} = \mu_{0} \vec{j}$ \\

      \hline
    \end{tabular}
  \end{table}
}
\end{center}

\begin{center}
{
  \begin{table}[H]
    \begin{tabular}{|l|c|c|}
      \hline
        \multicolumn{3}{|l|} {
          {\color{magenta}
           {\bf Static case within materials}
          }
        }\\
      \hline
      {\bf Gauss's law} &
        $\displaystyle \oint \vec{D} \cdot d\vec{S} =  \int \rho_{f} d\tau$ &
        $\displaystyle \vec{\nabla} \cdot \vec{D} = \rho_{f}$ \\

      {\bf Circuital law} &
        $\displaystyle \oint \vec{E} \cdot d\vec{\ell} = 0$ &
        $\displaystyle \vec{\nabla} \times \vec{E} = 0$ \\

      {\bf Gauss's law} (magn.) &
        $\displaystyle \oint \vec{B} \cdot d\vec{S} = 0$ &
        $\displaystyle \vec{\nabla} \cdot \vec{B} = 0$ \\

      {\bf Ampere's law} &
        $\displaystyle \oint \vec{H} \cdot d\vec{\ell} =  \int \vec{j}_{f} \cdot d\vec{S}$ &
        $\displaystyle \vec{\nabla} \times \vec{H} = \vec{j}_{f}$ \\
      \hline
    \end{tabular}
  \end{table}
}
\end{center}
}

\end{frame}


%
% Plan for this lecture
%

\begin{frame}{Plan for Lecture \thislecture}

In this lecture:
\begin{itemize}
   \item We will study how we need to {\bf extend Maxwell's eqs. in vaccum}
         in the case of {\bf time-dependent fields}.
\end{itemize}

\end{frame}

% ------------------------------------------------------------------------------
% ------------------------------------------------------------------------------

%
%
%

\begin{frame}{Time-dependent fields}

Having completed the study of Maxwell's equations for the static case (both in vaccum and in materials),
we will now turn our attention to the case where the electric and magnetic field can vary with time.

\begin{center}
   \includegraphics[width=0.95\textwidth]{./images/schematics/maxwell_eq_variations.png}\\
\end{center}

\end{frame}

%
%
%

\begin{frame}{A conductor moving in a magnetic field}

Consider a conductor with length L moves with velocity $\vec{u}$ inside a
homogenous magnetic field $\vec{B}$, as shown below:

\begin{center}
  \includegraphics[width=0.60\textwidth]{./images/schematics/conductor_in_magnetic_field_induced_emf.png}\\
\end{center}

\vspace{0.1cm}
Each electron in the conductor feels a magnetic force $\vec{F}_{M} = q \vec{u} \times \vec{B}$.\\
\vspace{0.1cm}
That magnetic force {\bf induces the build-up of charge}
which {\bf produces an electric field} $\vec{E}$:
Each electron feels an electric force $\vec{F}_{E} = q \vec{E}$.\\
\vspace{0.2cm}
The resulting {\bf electric force} $\vec{F}_{E}$ {\bf opposes the magnetic force} $\vec{F}_{M}$.

\end{frame}

%
%
%

\begin{frame}{A conductor moving in a magnetic field}

\begin{columns}
  \begin{column}{0.45\textwidth}
    \includegraphics[width=0.98\textwidth]{./images/schematics/conductor_in_magnetic_field_induced_emf.png}\\
  \end{column}
  \begin{column}{0.55\textwidth}
    At equilibrium, the electric and magnetic forces are equal in strength:
    \begin{equation*}
      \vec{F}_{E} =  \vec{F}_{M} \Rightarrow
       \cancel{q} \vec{E} = \cancel{q} \vec{u} \times \vec{B} \Rightarrow
     \end{equation*}
     \begin{equation*}
       \vec{E} = \vec{u} \times \vec{B}
     \end{equation*}
  \end{column}
\end{columns}

\vspace{0.4cm}

An electrical potential difference develops between the ends of the moving conductor,
which becomes a source of EMF:
\begin{equation*}
   \mathcal{E} = \int_{L} \vec{E} \cdot d\vec{\ell} = \int_{L} \Big( \vec{u} \times \vec{B} \Big) \cdot d\vec{\ell}
\end{equation*}

For a closed circuit:
\begin{equation*}
   \mathcal{E} = \oint_{L} \vec{E} \cdot d\vec{\ell} = \oint_{L} \Big( \vec{u} \times \vec{B} \Big) \cdot d\vec{\ell}
\end{equation*}

\end{frame}


%
%
%

\begin{frame}{A circuit moving through a magnetic field}

Now let's consider a simple rectangular circuit moving through a region that has a magnetic field $\vec{B}$:
\begin{center}
  \includegraphics[width=0.98\textwidth]{./images/schematics/circuit_moving_through_magnetic_field_all_3.png}\\
\end{center}

Conventions:
\begin{itemize}
  \item $\vec{B}$ is towards the page.
  \item Assuming current I flows anti-clockwise.
  \item Looping within the circuit: anti-clockwise
  \item Following the direction of the current I with our right-hand, the thumb points towards you -
        let's use that direction for the surface vector
\end{itemize}

\end{frame}

%
%
%

\begin{frame}{A circuit moving through a magnetic field}

\begin{columns}
  \begin{column}{0.30\textwidth}
    \includegraphics[width=0.99\textwidth]{./images/schematics/circuit_moving_through_magnetic_field_left.png}\\
  \end{column}
  \begin{column}{0.70\textwidth}
    Let's consider what happens as the circuit enters in the region of the magnetic field.
  \end{column}
\end{columns}

\vspace{0.3cm}

\begin{columns}[t]
  \begin{column}{0.50\textwidth}
  {\scriptsize
    \underline{EMF}:\\
    \vspace{0.1cm}
    \begin{itemize}
       \item Side `12':
             $\mathcal{E} = \int \Big( \vec{u} \times \vec{B} \Big) \cdot d\vec{\ell} = uBb$
       \item Side `34' ($\vec{B}=0$):
             $\mathcal{E} = 0$
       \item Side `23' and `14' ($\vec{u} \times \vec{B}$ $\perp$ $d\vec{\ell}$):
             $\mathcal{E} = 0$
       \item Summing-up the EMFs for all 4 sides:
            {\color{magenta}
              \begin{equation*}
               \mathcal{E} = \oint \vec{E} \cdot d\vec{\ell} = uBb
              \end{equation*}
            }
    \end{itemize}
  }
  \end{column}
  \begin{column}{0.50\textwidth}
  {\scriptsize
    \underline{Change in flux}:\\
    \begin{equation*}
      \frac{d\Phi_{M}}{dt} =
        \frac{d}{dt} \Big( \int \vec{B} \cdot d\vec{S} \Big) =
        \frac{d}{dt} \Big( \vec{B} \cdot \int d\vec{S} \Big) =
    \end{equation*}
    \begin{equation*}
          \frac{d}{dt} \Big( \vec{B} \cdot \vec{S} \Big) \xlongequal{\angle(\vec{B}, \vec{S}) = \pi}
          \frac{d}{dt} \Big( -BS \Big) \xlongequal{S=bx}
    \end{equation*}
    \begin{equation*}
        \frac{d}{dt} \Big( -Bbx \Big) =
        - \frac{dx}{dt} B b \Rightarrow
    \end{equation*}
    \begin{equation*}
        {\color{magenta}
          \frac{d\Phi_{M}}{dt} = -uBb
        }
    \end{equation*}
  }
  \end{column}
\end{columns}

\end{frame}

%
%
%

\begin{frame}{A circuit moving through a magnetic field}

\begin{columns}
  \begin{column}{0.30\textwidth}
    \includegraphics[width=0.99\textwidth]{./images/schematics/circuit_moving_through_magnetic_field_centre.png}\\
  \end{column}
  \begin{column}{0.70\textwidth}
    Let's consider what happens when the circuit is entirely within the region of the magnetic field.
  \end{column}
\end{columns}

\vspace{0.1cm}

\begin{columns}[t]
  \begin{column}{0.50\textwidth}
  {\scriptsize
    \underline{EMF}:\\
    \vspace{0.2cm}
    \begin{itemize}
       \item Side `12':
             $\mathcal{E} = \int \Big( \vec{u} \times \vec{B} \Big) \cdot d\vec{\ell} = uBb$
       \item Side `34':
             $\mathcal{E} = \int \Big( \vec{u} \times \vec{B} \Big) \cdot d\vec{\ell} = -uBb$\\
             (since $\vec{u} \times \vec{B}$ is same as for side `12',
             but $d\vec{\ell}$ has opposite direction).
       \item Side `23' and `14' ($\vec{u} \times \vec{B}$ $\perp$ $d\vec{\ell}$):
             $\mathcal{E} = 0$
       \item Summing-up the EMFs for all 4 sides:
            {\color{magenta}
              \begin{equation*}
               \mathcal{E} = \oint \vec{E} \cdot d\vec{\ell} = 0
              \end{equation*}
            }
    \end{itemize}
  }
  \end{column}
  \begin{column}{0.50\textwidth}
  {\scriptsize
    \underline{Change in flux}:\\
    \vspace{0.1cm}
    The circuit is entirely within the magnetic field so, as it moves,
    there are as many magnetic field line entering its surface as ones exiting.
    The rate of change of the magnetic flux is 0.
    {\color{magenta}
      \begin{equation*}
        \frac{d\Phi_{M}}{dt} = 0
      \end{equation*}
    }
  }
  \end{column}
\end{columns}

\end{frame}

%
%
%

\begin{frame}{A circuit moving through a magnetic field}

\begin{columns}
  \begin{column}{0.30\textwidth}
    \includegraphics[width=0.99\textwidth]{./images/schematics/circuit_moving_through_magnetic_field_right.png}\\
  \end{column}
  \begin{column}{0.70\textwidth}
    Finally, let's consider what happens as the circuit exits the region of the magnetic field.
  \end{column}
\end{columns}

\vspace{0.3cm}

\begin{columns}[t]
  \begin{column}{0.50\textwidth}
  {\scriptsize
    \underline{EMF}:\\
    \vspace{0.1cm}
    \begin{itemize}
       \item Side `12' ($\vec{B}=0$):
             $\mathcal{E} = 0$
       \item Side `34':
             $\mathcal{E} = \int \Big( \vec{u} \times \vec{B} \Big) \cdot d\vec{\ell} = -uBb$
       \item Side `23' and `14' ($\vec{u} \times \vec{B}$ $\perp$ $d\vec{\ell}$):
             $\mathcal{E} = 0$
       \item Summing-up the EMFs for all 4 sides:
            {\color{magenta}
              \begin{equation*}
               \mathcal{E} = \oint \vec{E} \cdot d\vec{\ell} = -uBb
              \end{equation*}
            }
    \end{itemize}
  }
  \end{column}
  \begin{column}{0.50\textwidth}
  {\scriptsize
    \underline{Change in flux}:\\
    \begin{equation*}
      \frac{d\Phi_{M}}{dt} =
        \frac{d}{dt} \Big( \int \vec{B} \cdot d\vec{S} \Big) =
        \frac{d}{dt} \Big( \vec{B} \cdot \int d\vec{S} \Big) =
    \end{equation*}
    \begin{equation*}
          \frac{d}{dt} \Big( \vec{B} \cdot \vec{S} \Big) \xlongequal{\angle(\vec{B}, \vec{S}) = \pi}
          \frac{d}{dt} \Big( -BS \Big) \xlongequal{S=bx}
    \end{equation*}
    \begin{equation*}
        \frac{d}{dt} \Big( -Bbx \Big) =
        - \frac{dx}{dt} B b = -(-u) B b \Rightarrow
    \end{equation*}
    \begin{equation*}
        {\color{magenta}
          \frac{d\Phi_{M}}{dt} = uBb
        }
    \end{equation*}
  }
  \end{column}
\end{columns}

\end{frame}

%
%
%

\begin{frame}{A circuit moving through a magnetic field}

A rectangular circuit moving through a region with magnetic field $\vec{B}$:
\begin{center}
  \includegraphics[width=0.88\textwidth]{./images/schematics/circuit_moving_through_magnetic_field_all_3.png}
\end{center}

\begin{columns}
  \begin{column}{0.55\textwidth}
  In summary:
  {\small
    \setlength{\extrarowheight}{10pt}
    \setlength{\arraycolsep}{5pt}
    \begin{table}[H]
        \begin{tabular}{|c||c|c|}
        \hline
               & $\displaystyle \oint_{L} \vec{E} \cdot d\vec{\ell}$ & $\displaystyle \frac{d\Phi_{M}}{dt}$\\
        \hline
          left   &  uBb &  -uBb \\
          centre & 0    &  0   \\
          right  &  -uBb & uBb \\
        \hline
        \end{tabular}
    \end{table}
  }
  \end{column}
  \begin{column}{0.45\textwidth}
     So, indeed, in all cases:
     \begin{equation*}
       \mathcal{E} = \oint_{L} \vec{E} \cdot d\vec{\ell} = - \frac{d\Phi_{M}}{dt}
     \end{equation*}
  \end{column}
\end{columns}

\end{frame}

%
%
%

\begin{frame}{Faraday's observations}

In 1831 Michael Faraday reported on a series of experiments.\\
\vspace{0.2cm}

A current flows in a wire loop when:
\vspace{0.2cm}
\begin{enumerate}[(a)]
{\small
  \item the loop is pulled through a magnetic field,
  \item the loop is at rest but the magnet moves in the opposite direction, and
  \item both the loop and the magnet are at rest but the strength of the magnetic field is varied
}
\end{enumerate}

\begin{center}
  \includegraphics[width=0.88\textwidth]{./images/schematics/faraday_law_schematic.png}
\end{center}

\end{frame}


%
%
%

\begin{frame}{Faraday's observations}

\begin{center}
  \includegraphics[width=0.88\textwidth]{./images/schematics/faraday_law_schematic.png}\\
\end{center}

\begin{itemize}
   \item In cases (a) and (b) it is the {\bf magnetic field} that is {\bf responsible for the EMF}
             (as in the example we studied earlier).
    \item But in case (c) both the loop and the magnet are stationary and the force felt
              by the electrons can not be magnetic.
              It is the {\bf electric field} that is {\bf responsible for the EMF}!
\end{itemize}

This led Faraday to realize that:
\begin{center}
{\bf A time-varying magnetic field induces an electric field}
\end{center}

\end{frame}

%
%
%

\begin{frame}{Faraday's law / Lenz's law}

In all cases the {\bf motional EMF} is directly related
to the {\bf change of the magnetic flux $\Phi_{M}$ though the circuit}:\\

\vspace{0.1cm}

\begin{columns} [T]
  \begin{column}{0.25\textwidth}
    \includegraphics[width=0.98\textwidth]{./images/people/faraday.png}\\
    {\tiny Michael Faraday (1791 - 1867).}
  \end{column}
  \begin{column}{0.75\textwidth}
   \begin{center}
    {\color{red}
    \begin{equation*}
      \mathcal{E} = \oint_{L} \vec{E} \cdot d\vec{\ell} = - \frac{d\Phi_{M}}{dt}
    \end{equation*}
    }
    \vspace{0.2cm}
    This is the so-called {\bf Faraday's law} (integral form).
   \end{center}
  \end{column}
\end{columns}

\vspace{0.1cm}

\begin{columns}
  \begin{column}{0.75\textwidth}
  {\scriptsize
    Lenz' law (1845): The EMF induced by a changing flux has a polarity
    such that the current flowing gives rise to a flux which opposes the change of flux.\\
    \vspace{0.1cm}
    The minus sign is a consequence of the {\bf conservation of energy} and
    of {\bf Newton's 3rd law} of motion: Induction is a an ``inertial reaction''.
    The system develops a current which tries to maintain the flux constant.\\
  }
  \end{column}
  \begin{column}{0.25\textwidth}
    \includegraphics[width=0.83\textwidth]{./images/people/lenz.jpg}\\
    {\tiny Heinrich Lenz (1804 - 1865).}
  \end{column}
\end{columns}

\end{frame}



%
%
%

\begin{frame}{Differential form of Faraday's law}

The integral form of Faraday's law is:
\begin{equation*}
   \oint_{L} \vec{E} \cdot d\vec{\ell} = - \frac{d}{dt} \int_{S} \vec{B} \cdot d\vec{S}
\end{equation*}

Using Stoke's theorem, we obtain its differential form:
\begin{equation*}
   \oint_{L} \vec{E} \cdot d\vec{\ell} = - \frac{d}{dt} \int_{S(L)} \vec{B} \cdot d\vec{S} \Rightarrow
   \int_{S(L)} \Big( \vec{\nabla} \times \vec{E} \Big) \cdot d\vec{S} =
     - \int_{S(L)} \frac{\partial \vec{B}}{\partial t} \cdot d\vec{S} \Rightarrow
\end{equation*}
\begin{equation*}
   \int_{S(L)} \Big( \vec{\nabla} \times \vec{E} + \frac{\partial \vec{B}}{\partial t} \Big) \cdot d\vec{S} = 0  \Rightarrow
   \vec{\nabla} \times \vec{E} + \frac{\partial \vec{B}}{\partial t} = 0 \Rightarrow
\end{equation*}
\begin{equation*}
{\color{red}
   \vec{\nabla} \times \vec{E} = - \frac{\partial \vec{B}}{\partial t}
}
\end{equation*}

\end{frame}



%
% Worked example
%

{
\problemslide

%
%
%

\begin{frame}{Worked example }

\begin{blockexmplque}{Question}
A circular wire loop on the x-y plane has a radius $r_0$ = 1 cm at time t = 0. \\
A homogeneous magnetic field of $3 \times 10^{-3}$ T in the positive z direction permeates the loop.
\begin{enumerate}
  \item Find the magnetic flux through the loop at t=0.
  \item The radius of the loop increases with $\displaystyle r(t) = r_0 + u \cdot t$
        with u = 1 cm/s. Find the magnetic flux through the loop at t = 4 s.
  \item Derive an expression for the EMF in the loop as a function of time.
\end{enumerate}
\end{blockexmplque}

\vspace{0.1cm}


The flux through the loop is given by  $\displaystyle \Phi = \int \vec{B} \cdot d\vec{S}$.

The loop is lying on the x-y plane, hence the surface vector
$d\vec{S}$ that is normal to the loop surface is along the z axis.
$\vec{B}$ is also along the z axis.

Therefore, the above dot product simplifies to $\displaystyle \Phi = \int B dS$.

\end{frame}

%
%
%

\begin{frame}{Worked example }

Since $\vec{B}$ is homogeneous:
\begin{equation*}
     \Phi = B \int  dS =  B \Big( \pi r^2\Big)
\end{equation*}
where $r$ is the radius of the loop.

\vspace{0.4cm}

In our case, r and hence $\Phi$ are functions of time:
\begin{equation*}
     \Phi(t) = B \Big( \pi r(t)^2\Big)
\end{equation*}

\vspace{0.3cm}

At time t = 0:
\begin{equation*}
   \Phi(t = 0) = B \Big( \pi r_0^2 \Big)
      = \Big( 3 \times 10^{-3} \; T \Big) \Big( 3.14  \cdot 0.01^2 \; m^2 \Big) = 9.4 \times 10^{-7} \; T \cdot m^2
\end{equation*}

\end{frame}

%
%
%

\begin{frame}{Worked example }

The radius of the loop increases as $\displaystyle r(t) = r_0 + u \cdot t$,
with u = 1 cm/s.

\vspace{0.3cm}

At t = 4 s, the radius of the loop is:
\begin{equation*}
     r(t = 4 \; s) = 0.01 \; cm + \Big(0.01 \; m/s \Big) \cdot \Big( 4 \; s \Big) = 0.05 \; m
\end{equation*}

\vspace{0.2cm}

Therefore the flux through the loop is now:
\begin{equation*}
     \Phi(4 \; s) = \Big( 3 \times 10^{-3} \; T \Big) \Big( 3.14  \cdot 0.05^2 \; m^2 \Big) = 2.35 \times 10^{-5} \; T \cdot m^2
\end{equation*}

\end{frame}

%
%
%

\begin{frame}{Worked example }

According to Faraday's law, the EMF $\displaystyle \mathcal{E} = \oint \vec{E} d\vec{\ell}$ developed in the wire loop
is equal to the negative rate of change of the magnetic flux through the loop:
\begin{equation*}
      \mathcal{E} = - \frac{d\Phi(t)}{dt} = - \frac{d}{dt} \bigg\{ B
      \Big( \pi r(t)^2\Big) \bigg\}  = - \pi B \frac{d}{dt} \bigg\{ r(t)^2 \bigg\}
\end{equation*}

Recall that $\displaystyle  r(t) = r_0 + u \cdot t$, therefore:
\begin{equation*}
      \mathcal{E}
          = - \pi B \frac{d}{dt} \bigg\{  \Big( r_0 + u \cdot t \Big)^2 \bigg\}
          = - \pi B \frac{d}{dt} \bigg\{  r_0^2 + 2 \cdot r_0 \cdot u \cdot t + u^2 \cdot t^2 \bigg\}
\end{equation*}
\begin{equation*}
      = - \pi B \Big(  2 \cdot r_0 \cdot u + 2\cdot u^2 \cdot t \Big)
      = - 2 \pi B u \Big( r_0 + u \cdot t \Big)
      = - 2 \pi B u r(t)
\end{equation*}

\end{frame}


} % Example


%
%
%

\begin{frame}{A problem with Ampere's law}

Consider a current I charging a parallel plate capacitor.\\
\vspace{0.3cm}
\begin{columns}
  \begin{column}{0.30\textwidth}
   \begin{center}
    \includegraphics[width=0.95\textwidth]{./images/schematics/problem_with_ampere_law.png}\\
   \end{center}
  \end{column}
  \begin{column}{0.70\textwidth}
  {\small
      Apply Ampere's law for the flat (light blue) surface at the top:
      \begin{equation*}
          \oint_{L} \vec{B} \cdot d\vec{\ell} = \mu_{0} I
     \end{equation*}
     So {\bf there is a magnetic field along L}.\\
     \vspace{0.2cm}
      Now, apply Ampere's law for the ``tophat'' (light green) surface
      made of the cylindrical part (no current flowing through it) and the
      circle between the two conducting plates (also no current):
      \begin{equation*}
          \oint_{L} \vec{B} \cdot d\vec{\ell} = 0
     \end{equation*}
     So {\bf there is NO magnetic field along L}.\\
   }
  \end{column}
\end{columns}

\vspace{0.3cm}

We get {\bf contradictory predictions} for the same path L! What is wrong?

\end{frame}

%
%
%

\begin{frame}{A problem with Ampere's Law}

We get {\bf contradictory predictions} for the same path L! What is wrong?\\
\vspace{0.3cm}

When we studied Faraday's law, we saw that a change in the magnetic field $\vec{B}$
is responsible for creating an electric field $\vec{E}$:
\begin{equation*}
  \vec{\nabla} \times \vec{E} = -\frac{\partial \vec{B}}{\partial t}
\end{equation*}
We should expect that the opposite may be true as well!
A change in the electric field $\vec{E}$ may modify the magnetic field $\vec{B}$.\\

\vspace{0.3cm}

In the example studied previously, the electric field $\vec{E}$ change with time.

\vspace{0.2cm}

How should we extend Ampere's law to take the effects of a changing electric field into account?

\end{frame}

%
%
%

\begin{frame}{Extending Ampere's law}

Let's start from the differential form of Ampere's law:
\begin{equation*}
  \vec{\nabla} \times \vec{B} = \mu_0 \vec{j}
\end{equation*}

We can calculate the divergence of both sides:
\begin{equation*}
  \vec{\nabla} \cdot \Big( \vec{\nabla} \times \vec{B} \Big) = \mu_0 \vec{\nabla} \cdot \vec{j}
\end{equation*}

The left-hand side is 0 (the divergence of the curl of a vector field is always 0).
So the right-hand side has to be 0 as well.
\begin{equation*}
  \vec{\nabla} \vec{j} = 0
\end{equation*}

But this is generally not true!
Recall the continuity equation that expresses the local conservation of charge:
\begin{equation*}
  \vec{\nabla} \cdot \vec{j} = -\frac{\partial \rho}{\partial t}
\end{equation*}

\end{frame}

%
%
%

\begin{frame}{Extending Ampere's law}

Ampere's law, as we know it so far, is at odds with the fundamental principle of
the local conservation of charge. Can we reconcile the two?

The charge density is given by Gauss' law
\begin{equation*}
  \vec{\nabla} \cdot \vec{E} = \frac{\rho}{\epsilon_0} \Rightarrow \rho =
   \epsilon_0 \vec{\nabla} \cdot \vec{E}
\end{equation*}

Taking the partial derivative of $\rho$ with respect to time:
\begin{equation*}
  \frac{\partial \rho}{\partial t} =
   \frac{\partial}{\partial t} \Big( \epsilon_0 \vec{\nabla} \cdot \vec{E} \Big) =
   \vec{\nabla} \cdot \Big( \epsilon_0 \frac{\partial \vec{E}}{\partial t} \Big)
\end{equation*}

Then, the continuity equation:
\begin{equation*}
  \vec{\nabla} \cdot \vec{j} = -\frac{\partial \rho}{\partial t}
\end{equation*}

becomes:
\begin{equation*}
  \vec{\nabla} \cdot \vec{j} =
    - \vec{\nabla} \cdot \Big( \epsilon_0 \frac{\partial \vec{E}}{\partial t} \Big)
\end{equation*}

\end{frame}


%
%
%

\begin{frame}{Extending Ampere's law}

We have written the continuity equation as:
\begin{equation*}
  \vec{\nabla} \cdot \vec{j} = - \vec{\nabla} \cdot \Big( \epsilon_0 \frac{\partial \vec{E}}{\partial t} \Big)
\end{equation*}
which suggests that:
\begin{equation*}
  \vec{\nabla} \cdot \Big( \vec{j} + \epsilon_0 \frac{\partial \vec{E}}{\partial t} \Big) = 0
\end{equation*}

{\bf This is interesting hint!}\\
Maxwell realised that all it takes to fix Ampere's law is to do the following substitution:
\begin{equation*}
  \vec{j} \rightarrow
  \vec{j} + \epsilon_0 \frac{\partial \vec{E}}{\partial t}
\end{equation*}

The term $\displaystyle \epsilon_0 \frac{\partial \vec{E}}{\partial t}$ is the density
of the so-called called {\bf displacement current}.

\end{frame}

%
%
%

\begin{frame}{Extending Ampere's law}

With this extension (adding the displacement current), Ampere's law is no longer at odds with the
local conservation of charge, as expressed with the continuity equation. Indeed:

\begin{equation*}
  \vec{\nabla} \times \vec{B} =
      \mu_0 \Big( \vec{j} + \epsilon_0 \frac{\partial \vec{E}}{\partial t} \Big)  \Rightarrow
\end{equation*}

\begin{equation*}
  \vec{\nabla} \cdot \Big( \vec{\nabla} \times \vec{B} \Big) =
    \vec{\nabla} \cdot \Big( \mu_0 \vec{j} + \epsilon_0 \mu_0 \frac{\partial \vec{E}}{\partial t} \Big)  \Rightarrow
\end{equation*}

\begin{equation*}
  0 =
    \cancel{\mu_0} \vec{\nabla} \cdot \vec{j} + \epsilon_0 \cancel{\mu_0} \frac{\partial}{\partial t}
    \Big( \vec{\nabla} \cdot  \vec{E}\Big)  \Rightarrow
\end{equation*}

\begin{equation*}
  0 =
    \vec{\nabla} \cdot \vec{j} + \cancel{\epsilon_0} \frac{\partial}{\partial t}
      \Big( \frac{\rho}{\cancel{\epsilon_0}} \Big)  \Rightarrow
\end{equation*}

\begin{equation*}
    \frac{\partial \rho}{\partial t} = - \vec{\nabla} \cdot \vec{j}
\end{equation*}

\end{frame}


%
% Worked example
%

{
\problemslide

%
%
%

\begin{frame}{Worked example }

\begin{blockexmplque}{Question}
\begin{columns}
  \begin{column}{0.22\textwidth}
   \begin{center}
     \includegraphics[width=0.90\textwidth]{./images/problems/lect6_capacitor.png}
   \end{center}
  \end{column}
  \begin{column}{0.78\textwidth}
     A parallel-plate capacitor with circular plates of radius R is being
     charged as shown in the figure on the left.\\
     Derive an expression for the magnetic field at radius r (in
     the volume within the two plates) for the case r $\leq$ R.
  \end{column}
\end{columns}
\end{blockexmplque}

\vspace{0.2cm}

A magnetic field can be created either by a current or a changing electric field.
This is expressed by Ampere's law:
\begin{equation*}
    \oint_{L} \vec{B} \cdot d\vec{\ell} = \mu_0 \int_{S} \Big( \vec{j} + \epsilon_0  \frac{\partial \vec{E}}{\partial t} \Big) \cdot d\vec{S}
\end{equation*}

\end{frame}

%
%
%

\begin{frame}{Worked example }

There is no current between the capacitor plates, but the electric flux there is changing.
Thus, Ampere's law reduces to:
\begin{equation*}
    \oint_{L} \vec{B} \cdot d\vec{\ell} = \mu_0 \epsilon_0 \int_{S} \Big( \frac{\partial \vec{E}}{\partial t} \Big) \cdot d\vec{S}
\end{equation*}

\vspace{0.2cm}

We choose a circular Amperian loop with radius r $\le$ R.

\vspace{0.2cm}

The magnetic field $\vec{B}$ at all points along the loop is tangent to the loop, as is the path element $d\vec{\ell}$.
Thus $\vec{B}$ and $d\vec{\ell}$ are parallel or antiparallel at each point of the loop.

\vspace{0.2cm}

For simplicity, assume that we step along the loop in such a way so that $\vec{B}$ and $d\vec{\ell}$ are parallel.
Therefore:
\begin{equation*}
    \oint_{L} \vec{B} \cdot d\vec{\ell} = \oint_{L}  B \; d\ell
\end{equation*}

\end{frame}

%
%
%

\begin{frame}{Worked example }

Due to the circular symmetry of the plates, we can also assume that $\vec{B}$ has the same
magnitude at every point around the loop. Thus, B can be taken outside the integral.

\vspace{0.2cm}

The integral that remains is $\oint_{L}  d\ell$ which simply gives the circumference $2 \pi r$ of the loop.
Therefore, the integral becomes:
\begin{equation*}
    \oint_{L} \vec{B} \cdot d\vec{\ell} = B \; \Big( 2 \pi r\Big)
\end{equation*}

\vspace{0.2cm}

Substituting the above result into Ampere's law gives:
\begin{equation*}
    B \; \Big( 2 \pi r\Big) = \mu_0 \epsilon_0 \int_{S} \Big( \frac{\partial \vec{E}}{\partial t} \Big) \cdot d\vec{S} \Rightarrow
    B = \frac{\mu_0 \epsilon_0}{2 \pi r} \int_{S} \Big( \frac{\partial \vec{E}}{\partial t} \Big) \cdot d\vec{S} \Rightarrow
\end{equation*}
\begin{equation*}
    B = \frac{\mu_0 \epsilon_0}{2 \pi r}  \frac{\partial}{\partial t} \int_{S} \vec{E} \cdot d\vec{S}
\end{equation*}

\end{frame}

%
%
%

\begin{frame}{Worked example }

We assume that the electric field $\vec{E}$ is uniform between the
capacitor plates and directed perpendicular to the plates.
The electric flux through the Amperian loop is simply $EA$,
where $A$ is the (constant) area encircled by the loop within the electric field.
The previous equation becomes:

\begin{equation*}
    B = \frac{\mu_0 \epsilon_0}{2 \pi r}  \frac{d}{dt} \Big( E A \Big)
       = \frac{\mu_0 \epsilon_0}{2 \pi r}  A \frac{dE}{dt}
\end{equation*}

\vspace{0.2cm}

The area A that is encircled by the Amperial loop within the electric field is
the full area $\pi r^2$ of the loop because the loop's radius r is less than (or equal to)
the plate radius R.   Substituting $\pi r^2$ for $A$, we have:
\begin{equation*}
    B = \frac{\mu_0 \epsilon_0}{2 \pi r}  \pi r^2 \frac{dE}{dt} \Rightarrow
    B = \frac{\mu_0 \epsilon_0 r}{2} \frac{dE}{dt}
\end{equation*}

\end{frame}

%
%
%

\begin{frame}{Worked example }

Ignoring edge effects,
the electric field E is uniform within the plates of the parallel plate capacitor
and vanishes outside the volume within the two plates.
Using a cylindrical Gaussian surface whose bases are parallel to the
plates and which encloses the upper plate, we have:
\begin{equation*}
    E \pi R^2 = \frac{Q}{\epsilon_0} \Rightarrow
    E = \frac{1}{\pi R^2 \epsilon_0} Q
\end{equation*}

Therefore, the rate of change of E is given by:
\begin{equation*}
    \frac{dE}{dt} = \frac{1}{\pi R^2 \epsilon_0} \frac{dQ}{dt}
    \xRightarrow {I = dQ/dt}
    \frac{dE}{dt} = \frac{1}{\pi R^2 \epsilon_0} I
\end{equation*}

Substituting the expression for $dE/dt$ in the expression we obtained
earlier for B, we have:
\begin{equation*}
    B = \frac{\mu_0 \epsilon_0 r}{2} \frac{1}{\pi R^2 \epsilon_0} I
    \Rightarrow
    B = \frac{\mu_0 I r}{2 \pi R^2}
\end{equation*}

This is the required expression for the magnetic field at radius r $\le$ R.

\end{frame}



} % end worked example


% ------------------------------------------------------------------------------
% ------------------------------------------------------------------------------

%
% What to remember
%

\renewcommand{\lecturesummarytitle}{Main points to remember }

\renewcommand{\summarizedlecture}{8 }


%
%
%

\begin{frame}{Lecture \summarizedlecture revision (Conductors moving in a magnetic field)}

\begin{columns}
  \begin{column}{0.50\textwidth}
     We consider a conductor with length L moves with velocity $\vec{u}$ inside a
     homogenous magnetic field $\vec{B}$, as shown on the right.
  \end{column}
  \begin{column}{0.50\textwidth}
   \begin{center}
     \includegraphics[width=0.90\textwidth]{./images/schematics/conductor_in_magnetic_field_induced_emf.png}\\
   \end{center}
  \end{column}
\end{columns}

\vspace{0.2cm}

Each electron in the conductor feels a magnetic force $\vec{F}_{M} = q \vec{u} \times \vec{B}$.\\
\vspace{0.1cm}
That magnetic force {\bf induces the build-up of charge}
which {\bf produces an electric field} $\vec{E}$:
Each electron feels an electric force $\vec{F}_{E} = q \vec{E}$.\\
\vspace{0.2cm}
The resulting {\bf electric force} $\vec{F}_{E}$ {\bf opposes the magnetic force} $\vec{F}_{M}$.\\

\vspace{0.2cm}

An electrical potential difference develops between the ends of the moving conductor,
which becomes a source of EMF:
\begin{equation*}
   \mathcal{E} = \int_{L} \vec{E} \cdot d\vec{\ell} = \int_{L} \Big( \vec{u} \times \vec{B} \Big) \cdot d\vec{\ell}
\end{equation*}

\end{frame}

%
%
%

\begin{frame}{Lecture \summarizedlecture revision  (Circuit moving in a magnetic field)}

A rectangular circuit moving through a region with magnetic field $\vec{B}$:
\begin{center}
  \includegraphics[width=0.88\textwidth]{./images/schematics/circuit_moving_through_magnetic_field_all_3.png}
\end{center}

\begin{columns}
  \begin{column}{0.55\textwidth}
  In summary:
  {\small
    \setlength{\extrarowheight}{10pt}
    \setlength{\arraycolsep}{5pt}
    \begin{table}[H]
        \begin{tabular}{|c||c|c|}
        \hline
               & $\displaystyle \oint_{L} \vec{E} \cdot d\vec{\ell}$ & $\displaystyle \frac{d\Phi_{M}}{dt}$\\
        \hline
          left   &  uBb &  -uBb \\
          centre & 0    &  0   \\
          right  &  -uBb & uBb \\
        \hline
        \end{tabular}
    \end{table}
  }
  \end{column}
  \begin{column}{0.45\textwidth}
     So, indeed, in all cases:
     \begin{equation*}
       \mathcal{E} = \oint_{L} \vec{E} \cdot d\vec{\ell} = - \frac{d\Phi_{M}}{dt}
     \end{equation*}
  \end{column}
\end{columns}

\end{frame}

%
%
%

\begin{frame}{Lecture \summarizedlecture revision (Faraday's observations)}

In 1831 Michael Faraday reported on a series of experiments.\\
\vspace{0.2cm}

A current flows in a wire loop when:
\vspace{0.1cm}
\begin{enumerate}[(a)]
{\small
  \item the loop is pulled through a magnetic field,
  \item the loop is at rest but the magnet moves in the opposite direction, and
  \item both the loop and the magnet are at rest but the strength of the magnetic field is varied
}
\end{enumerate}

\begin{center}
  \includegraphics[width=0.80\textwidth]{./images/schematics/faraday_law_schematic_with_notes.png}
\end{center}
This led Faraday to realize that:
\begin{center}
{\bf A time-varying magnetic field induces an electic field}
\end{center}

\end{frame}

%
%
%

\begin{frame}{Lecture \summarizedlecture revision (Faraday's law / Lenz's law)}

In all cases the {\bf motional EMF} is directly related
to the {\bf change of the magnetic flux $\Phi_{M}$ though the circuit}:\\

\vspace{0.1cm}

\begin{columns} [T]
  \begin{column}{0.25\textwidth}
    \includegraphics[width=0.98\textwidth]{./images/people/faraday.png}\\
    {\tiny Michael Faraday (1791 - 1867).}
  \end{column}
  \begin{column}{0.75\textwidth}
   \begin{center}
    \begin{equation*}
      \mathcal{E} = \oint_{L} \vec{E} \cdot d\vec{\ell} = - \frac{d\Phi_{M}}{dt}
    \end{equation*}
    \vspace{0.2cm}
    This is the so-called {\bf Faraday's law}.
   \end{center}
  \end{column}
\end{columns}

\vspace{0.1cm}

\begin{columns}
  \begin{column}{0.75\textwidth}
  {\scriptsize
    Lenz' law (1845): The EMF induced by a changing flux has a polarity
    such that the current flowing gives rise to a flux which opposes the change of flux.\\
    \vspace{0.1cm}
    The minus sign is a consequence of the {\bf conservation of energy} and
    of {\bf Newton's 3rd law} of motion: Induction is a an ``inertial reaction''.
    The system develops a current which tries to maintain the flux constant.\\
  }
  \end{column}
  \begin{column}{0.25\textwidth}
    \includegraphics[width=0.83\textwidth]{./images/people/lenz.jpg}\\
    {\tiny Heinrich Lenz (1804 - 1865).}
  \end{column}
\end{columns}

\end{frame}

%
%
%

\begin{frame}{Lecture \summarizedlecture revision (Maxwell correction in Ampere's law)}

We also studied a case where Ampere's law led to paradoxical results.\\

\vspace{0.2cm}

We also saw that Ampere's law (as we knew it) was inconsistent with
the continuity equation (which expresses the local conservation of charge).\\

\vspace{0.2cm}

The problem of course was that we took a law from magnetostatics and
applied it in a different context (electrodynamics) where it is no longer valid.

\vspace{0.2cm}

Maxwell realised that all it takes to fix Ampere's law is to do the following substitution:
\begin{equation*}
  \vec{j} \rightarrow
  \vec{j} + \epsilon_0 \frac{\partial \vec{E}}{\partial t}
\end{equation*}

The term $\displaystyle \epsilon_0 \frac{\partial \vec{E}}{\partial t}$ is the
density of the so-called {\bf displacement current}.

\end{frame}

%
%
%

\begin{frame}{Lecture \summarizedlecture revision (Maxwell's eqs for the dynamic case)}

Compared with what we had seen in the study of electrostatics and magnetostatics,
the study of time-dependent fields (electrodynamics) brought the following complication:\\

\vspace{0.1cm}

\begin{itemize}
   \item {\bf Electric fields are produced} not only by electric charges,
             but also {\bf by changing magnetic fields!}
     \begin{equation*}
        \vec{\nabla} \times \vec{E} = {\color{red} - \frac{\partial \vec{B}}{\partial t}}
     \end{equation*}

   \item {\bf Magnetic fields are produced} not only by electric currents,
             but also {\bf by changing electric fields!}
     \begin{equation*}
         \vec{\nabla} \times \vec{B} = \mu_{0} \Big( \vec{j} + {\color{red} \epsilon_0 \frac{\partial \vec{E}}{\partial t} } \Big)
     \end{equation*}
\end{itemize}

{\small
The full list of Maxwell equations for the static and dynamic cases in vacuum is shown on the next slide.
Notice that all 4 equations are coupled in the dynamic case.\\
}

\end{frame}

%
%
%

\begin{frame}{Lecture \summarizedlecture revision (Maxwell's eqs. for the dynamic case)}

{\small

\begin{center}
{
  \begin{table}[H]
    \begin{tabular}{|l|c|c|}
      \hline
        \multicolumn{3}{|l|} {
          {\color{magenta}
           {\bf Static case (in vacuum)}
          }
        }\\
      \hline
      {\bf Gauss's law} &
        $\displaystyle \oint \vec{E} \cdot d\vec{S} = \frac{1}{\epsilon_0} \int \rho d\tau$ &
        $\displaystyle \vec{\nabla} \cdot \vec{E} = \frac{\rho}{\epsilon_0}$ \\

      {\bf Circuital law} &
        $\displaystyle \oint \vec{E} \cdot d\vec{\ell} = 0$ &
        $\displaystyle \vec{\nabla} \times \vec{E} = 0$ \\

      {\bf Gauss's law} (magn) &
        $\displaystyle  \oint \vec{B} \cdot d\vec{S} = 0$ &
        $\displaystyle  \vec{\nabla} \cdot \vec{B} = 0$ \\

      {\bf Ampere's law} &
        $\displaystyle \oint \vec{B} \cdot d\vec{\ell} = \mu_{0} \int \vec{j} \cdot d\vec{S}$ &
        $\displaystyle \vec{\nabla} \times \vec{B} = \mu_{0} \vec{j}$ \\
      \hline
    \end{tabular}
  \end{table}
}
\end{center}


\begin{center}
{
  \begin{table}[H]
    \begin{tabular}{|l|c|c|}
      \hline
        \multicolumn{3}{|l|} {
          {\color{magenta}
           {\bf Generalization of above for the dynamic case (in vacuum)}
          }
        }\\
      \hline
      {\bf Gauss's law} &
        $\displaystyle \oint \vec{E} \cdot d\vec{S} = \frac{1}{\epsilon_0} \int \rho d\tau$ &
        $\displaystyle \vec{\nabla} \cdot \vec{E} = \frac{\rho}{\epsilon_0}$ \\

      {\bf Faraday's law} &
        $\displaystyle \oint \vec{E} \cdot d\vec{\ell} =  -\frac{\partial}{\partial t} \int \vec{B} \cdot d\vec{S}$ &
        $\displaystyle \vec{\nabla} \times \vec{E} = -  \frac{\partial \vec{B}}{\partial t}$ \\

      {\bf Gauss's law} (magn) &
        $\displaystyle  \oint \vec{B} \cdot d\vec{S} = 0$ &
        $\displaystyle  \vec{\nabla} \cdot \vec{B} = 0$ \\

      {\bf Ampere's law} &
        $\displaystyle \oint \vec{B} \cdot d\vec{\ell} =
           \mu_{0} \int \Big( \vec{j} + \epsilon_0 \frac{\partial \vec{E}}{\partial t}\Big) \cdot d\vec{S}$ &
        $\displaystyle \vec{\nabla} \times \vec{B} =
           \mu_{0} \Big( \vec{j} + \epsilon_0 \frac{\partial \vec{E}}{\partial t}\Big)$ \\
      \hline
    \end{tabular}
  \end{table}
}
\end{center}

}
\end{frame}



%
%
%

\begin{frame}{Lecture \summarizedlecture - \lecturesummarytitle}

Compared with what we had seen in the study of electrostatics and magnetostatics,
the study of time-dependent fields (electrodynamics) brought the following complication:\\

\vspace{0.1cm}

\begin{itemize}
   \item {\bf Electric fields are produced} not only by electric charges,
             but also {\bf by changing magnetic fields!}
     \begin{equation*}
        \vec{\nabla} \times \vec{E} = {\color{red} - \frac{\partial \vec{B}}{\partial t}}
     \end{equation*}

   \item {\bf Magnetic fields are produced} not only by electric currents,
             but also {\bf by changing electric fields!}
     \begin{equation*}
         \vec{\nabla} \times \vec{B} = \mu_{0} \Big( \vec{j} + {\color{red} \epsilon_0 \frac{\partial \vec{E}}{\partial t} } \Big)
     \end{equation*}
\end{itemize}

{\small
The full list of Maxwell equations for the static and dynamic cases in vacuum is shown on the next slide.
Notice that whereas the 2 equations involving $\vec{E}$ and the 2 equations involving $\vec{B}$ were decoupled in the static case,
all 4 equations are coupled in the dynamic case.\\
}

\end{frame}


% %
% %
% %
%
% \begin{frame}{Maxwell's equations for the static and dynamic cases}
%
% {\small
%
% \begin{center}
% {
%   \begin{table}[H]
%     \begin{tabular}{|l|c|c|}
%       \hline
%         \multicolumn{3}{|l|} {
%           {\color{magenta}
%            {\bf Static case (in vacuum)}
%           }
%         }\\
%       \hline
%       {\bf Gauss's law} &
%         $\displaystyle \oint \vec{E} d\vec{S} = \frac{1}{\epsilon_0} \int \rho d\tau$ &
%         $\displaystyle \vec{\nabla} \cdot \vec{E} = \frac{\rho}{\epsilon_0}$ \\
%
%       {\bf Circuital law} &
%         $\displaystyle \oint \vec{E} d\vec{\ell} = 0$ &
%         $\displaystyle \vec{\nabla} \times \vec{E} = 0$ \\
%
%       no magn. monopoles' &
%         $\displaystyle  \oint \vec{B} d\vec{S} = 0$ &
%         $\displaystyle  \vec{\nabla} \cdot \vec{B} = 0$ \\
%
%       {\bf Ampere's law} &
%         $\displaystyle \oint \vec{B} d\vec{\ell} = \mu_{0} \int \vec{j} d\vec{S}$ &
%         $\displaystyle \vec{\nabla} \times \vec{B} = \mu_{0} \vec{j}$ \\
%       \hline
%     \end{tabular}
%   \end{table}
% }
% \end{center}
%
%
% \begin{center}
% {
%   \begin{table}[H]
%     \begin{tabular}{|l|c|c|}
%       \hline
%         \multicolumn{3}{|l|} {
%           {\color{magenta}
%            {\bf Generalization of above for the dynamic case (in vacuum)}
%           }
%         }\\
%       \hline
%       {\bf Gauss's law} &
%         $\displaystyle \oint \vec{E} d\vec{S} = \frac{1}{\epsilon_0} \int \rho d\tau$ &
%         $\displaystyle \vec{\nabla} \cdot \vec{E} = \frac{\rho}{\epsilon_0}$ \\
%
%       {\bf Circuital law} &
%         $\displaystyle \oint \vec{E} d\vec{\ell} =  -\frac{\partial}{\partial t} \int \vec{B} d\vec{S}$ &
%         $\displaystyle \vec{\nabla} \times \vec{E} = -  \frac{\partial \vec{B}}{\partial t}$ \\
%
%       no magn. monopoles &
%         $\displaystyle  \oint \vec{B} d\vec{S} = 0$ &
%         $\displaystyle  \vec{\nabla} \cdot \vec{B} = 0$ \\
%
%       {\bf Ampere's law} &
%         $\displaystyle \oint \vec{B} d\vec{\ell} = \mu_{0} \int \Big( \vec{j} + \epsilon_0 \frac{\partial \vec{E}}{\partial t}\Big) d\vec{S}$ &
%         $\displaystyle \vec{\nabla} \times \vec{B} = \mu_{0} \Big( \vec{j} + \epsilon_0 \frac{\partial \vec{E}}{\partial t}\Big)$ \\
%       \hline
%     \end{tabular}
%   \end{table}
% }
% \end{center}
%
% }
% \end{frame}


%
% Plan for the next lecture
%

\begin{frame}{At the next lecture (Lecture \nextlecture)}

\begin{itemize}
  \item We will study the solutions of the time-dependent Maxwell equations in vaccum
            away from sources ($\rho = 0$, $\vec{j} = \vec{0}$)
  \vspace{0.3cm}
  \item {\bf Light}!
\end{itemize}

\vspace{0.3cm}

\begin{center}
   \includegraphics[width=0.55\textwidth]{./images/photos/beam_of_light.jpg}\\
\end{center}

\end{frame}

%
% Optional reading
%

\begin{frame}[plain,c]
\begin{center}
{\Huge \bf Optional reading for Lecture \thislecture}
\end{center}
\end{frame}

% ------------------------------------------------------------------------------

%
% Worked example :
%

{
\problemslide

%
%
%

\begin{frame}{Worked example: Wire falling in magnetic field}

  \begin{blockexmplque}{Question}
    \begin{minipage}[l]{0.22\textwidth}
     \begin{center}
         \includegraphics[width=0.90\textwidth]{./images/problems/lect08_wire_falls_in_b_field}
     \end{center}
    \end{minipage}
    \begin{minipage}[r]{0.75\textwidth}
       A long straight wire parallel to the $y$ axis lies in a uniform magnetic
       field $\vec{B} = B \hat{x}$, as shown in the figure on the left.
       The mass per unit length and resistance per unit length of the wire
       are $\lambda_{m}$ and $\lambda_{r}$ respectively.\\
    \end{minipage}
    \vspace{0.1cm}
    The wire may be considered to extend to the edges of the field,
    where the ends are connected to one another by a massless perfect conductor
    which lies outside the field. Fringing effects can be neglected.\\
    The wire is allowed to fall under the influence of gravity $(\vec{g} = -g \hat{z})$.\\
    Find:
    \begin{itemize}
    \item the electromotive force developed across the wire,
    \item the current flowing along the wire,
    \item the magnetic force acting on the wire, and
    \item the terminal velocity of the wire as it falls through the magnetic field.
    \end{itemize}
  \end{blockexmplque}

\end{frame}

%
%
%

\begin{frame}{Worked example: Wire falling in magnetic field}

  As the wire falls within the magnetic field with a velocity $u$,
  the EMF $\varepsilon$ developed across the wire is given by:
  \begin{equation*}
    \varepsilon = \int (\vec{u} \times \vec{B}) \cdot d\vec{\ell}
                = uB\ell
  \end{equation*}
  where $\ell$ is the length of the wire
  (which is fully within the magnetic field).

  The resistance $R$ of the wire of length $\ell$ is:
  \begin{equation*}
    R = \lambda_{r} \ell
  \end{equation*}

  Therefore, the current that will flow in the wire is given by:
  \begin{equation*}
    I = \frac{\varepsilon}{R}
      = \frac{uB\cancel{\ell}}{\lambda_{r}\cancel{\ell}}
      = \frac{uB}{\lambda_{r}}
  \end{equation*}
  This current is flowing towards the $-\hat{y}$ direction.

\end{frame}

%
%
%

\begin{frame}{Worked example: Wire falling in magnetic field}

  The force exerted on the current-carrying wire because of the
  magnetic field is:
  \begin{equation*}
    \vec{F}_{B} =
     I \int d\vec{\ell} \times \vec{B}
  \end{equation*}

  Substituting $I$ from the previous expression, and considering
  $d\vec{\ell}$ to be in the direction of the current, we have:
  \begin{equation*}
    \vec{F}_{B} =
     \frac{uB}{\lambda_{r}} \int \Big( dy(-\hat{y})\Big) \times \Big( B \hat{x}\Big) =
     \frac{uB^2}{\lambda_{r}} \Big(\int dy \Big) \Big(- \hat{y}) \times \hat{x}\Big) \Rightarrow
  \end{equation*}
  \begin{equation*}
    \vec{F}_{B} =
     \frac{uB^2\ell}{\lambda_{r}} \hat{z}
  \end{equation*}
  This force will oppose the downwards gravitational force $\vec{F}_{g}$.

\end{frame}

%
%
%

\begin{frame}{Worked example: Wire falling in magnetic field}

  The wire carries a mass given by:
  \begin{equation*}
     m = \lambda_{m} \ell
  \end{equation*}
  and, therefore, the force of gravity is:
  \begin{equation*}
     \vec{F}_{g} = m \vec{g} = - \lambda_{m} \ell g \hat{z}
  \end{equation*}

  The wire will reach its terminal velocity $u^{\prime}$ when the
  two forces cancel each other out exactly.
  Therefore, $u^{\prime}$ is given by the condition:
  \begin{equation*}
     \frac{u^{\prime} B^2 \cancel{\ell}}{\lambda_{r}} = \lambda_{m} \cancel{\ell} g
  \end{equation*}

  Solving for $u^{\prime}$ we find:
  \begin{equation*}
     u^{\prime} = \frac{\lambda_{r} \lambda_{m} g}{B^2}
  \end{equation*}

\end{frame}

} % Worked example

% ------------------------------------------------------------------------------

%
% Worked example :
%

{
\problemslide

%
%
%

\begin{frame}{Worked example: Displacement current in capacitor}

  \begin{blockexmplque}{Question}
    At what rate must the potential difference between the plates
    of a parallel plate capacitor with a 2.0 $\mu$F capacitance
    be changed to produce a displacement current of 1.5 A?
  \end{blockexmplque}

  Let the area plate be A and the plate separation be d.
  The displacement current $I_{d}$ is:

  \begin{equation*}
    I_{d} = \epsilon_0 \frac{d\Phi_{E}}{dt}
    \xRightarrow{\Phi_{E} = EA}
    I_{d} = \epsilon_0 \frac{d}{dt} (EA)
          = \epsilon_0 A \frac{dE}{dt}
  \end{equation*}

  \begin{equation*}
    \xRightarrow{E = V/d}
    I_{d} = \epsilon_0 A \frac{d}{dt} (\frac{V}{d})
          = \frac{\epsilon_0 A}{d} \frac{dV}{dt}
    \xRightarrow{C = \epsilon_0 A/d}
    I_{d} = C \frac{dV}{dt}
  \end{equation*}

  \begin{equation*}
    \Rightarrow
     \frac{dV}{dt} = \frac{I_{d}}{C}
                   = \frac{15.0 \; A}{2 \times 10^{-6} \; F}
                   = 7.5 \times 10^{5} \; V/s
  \end{equation*}

\end{frame}

} % Worked example

% ------------------------------------------------------------------------------

%
% Worked example :
%

{
\problemslide

%
%
%

\begin{frame}{Worked example: Induced current in rectangular wire loop}

  \begin{blockexmplque}{Question}

    The figure below shows a wire that forms a rectangle (W = 20 cm, H = 30 cm)
    and has a resistance of 5.0 m$\Omega$. Its interior is split into three
    equal areas, with magnetic fields $\vec{B}_{1}$, $\vec{B}_{2}$ and $\vec{B}_{3}$.
    The fields are uniform within each region and directly
    out of or into the page as indicated.

    The graph below gives the change in the z components (B$_{z}$)
    of the three fields with time t;
    the vertical axis scale is set by
    B$_{s}$ = 4.0 $\mu$T and B$_{b}$ = 2.5B$_{s}$
    and the horizontal axis scale is set by t$_{s}$ = 2.0 s.

    \begin{minipage}[r]{0.30\textwidth}
      What are the
      \begin{itemize}
        \item magnitude, and
        \item direction of the current induced in the wire?
      \end{itemize}
    \end{minipage}
    \begin{minipage}[l]{0.30\textwidth}
     \begin{center}
    	 \includegraphics[width=0.95\textwidth]{./images/problems/lect08_rectangular_wire_3bfield_regions}\\
     \end{center}
    \end{minipage}
    \begin{minipage}[l]{0.38\textwidth}
     \begin{center}
    	 \includegraphics[width=0.95\textwidth]{./images/problems/lect08_rectangular_wire_3bfield_regions_Bz}\\
     \end{center}
    \end{minipage}

  \end{blockexmplque}

\end{frame}

%
%
%

\begin{frame}{Worked example: Induced current in rectangular wire loop}

  \vspace{-0.2cm}

  The induced emf is:
  \begin{equation*}
     \varepsilon = - \sum_{i} \frac{d\Phi_{B;i}}{dt}
  \end{equation*}

  If the surface vector of the loop is collinear with
  $\vec{B}_1$ and $\vec{B}_2$, the flux due to $\vec{B}_1$ and $\vec{B}_2$
  is positive, whereas the flux due to $\vec{B}_3$ is negative. Therefore:
  \begin{equation*}
     \varepsilon = \frac{1}{3} H W
      \Big\{ - \frac{dB_1}{dt} - \frac{dB_2}{dt} + \frac{dB_3}{dt} \Big\} =
  \end{equation*}
  \begin{equation*}
     \frac{(0.30 \; m) (0.20 \; m)}{3}
                     \Big\{ - \frac{4  \times 10^{-6} \; T}{2.0 \; s}
                            - \frac{2  \times 10^{-6} \; T}{2.0 \; s}
                            + \frac{10 \times 10^{-6} \; T}{2.0 \; s} \Big\}
     = 4 \times 10^{-8} \; V
  \end{equation*}

  The plus sign means that the emf is dominated by changes in $\vec{B}_3$.

  The current induced by $\varepsilon$ is:
  \begin{equation*}
    I = \frac{|\varepsilon|}{R} = \frac{4 \times 10^{-8} \; V}{5 \times 10^{-3} \; \Omega}
      \approx 8 \; {\mu}A
  \end{equation*}

  By Lenz's law, the induced emf (and current) resist to these changes in the magnetic flux.
  Therefore, the direction of the current is counter-clockwise.

\end{frame}

} % Worked example

% ------------------------------------------------------------------------------

%
% Worked example :
%

{
\problemslide

%
%
%

\begin{frame}{Worked example: Loop in uniform time-dependent $\vec{B}$}

  \begin{blockexmplque}{Question}

    \begin{minipage}[l]{0.30\textwidth}
     \begin{center}
    	 \includegraphics[width=0.98\textwidth]{./images/problems/lect08_square_loop_with_emf_partially_within_bfield}\\
     \end{center}
    \end{minipage}
    \begin{minipage}[r]{0.69\textwidth}
      A square wire loop with 2.00 m sides is perpendicular to a uniform
      magnetic field, with half the area of the loop in the field,
       as shown in the figure on the left.

      The loop contains an ideal battery with emf $\varepsilon_{bat}$ = 20 V.
      If the magnitude of the field varies with time according
      to $B$ = 0.0420 - 0.870$t$,
      with $B$ in Teslas and $t$ in seconds, what are the
      \begin{itemize}
        \item net emf in the circuit, and
        \item the direction of the (net) emf around the loop?
      \end{itemize}
    \end{minipage}

  \end{blockexmplque}

  Let $L$ be the length of a side of the square circuit.
  Then the magnetic flux through the circuit is:
  \begin{equation*}
    \Phi_B = \frac{1}{2} L^2 B
  \end{equation*}

\end{frame}

%
%
%

\begin{frame}{Worked example: Loop in uniform time-dependent $\vec{B}$}

  The induced emf is:
  \begin{equation*}
    \varepsilon_{induced} = - \frac{d\Phi_{B}}{dt} = - \frac{1}{2} L^2 \frac{dB}{dt}
  \end{equation*}

  The rate of change of the given field is:
  \begin{equation*}
    \frac{dB}{dt} = \frac{d}{dt} (0.0420 - 0.870t) = -0.870 \; T/s
  \end{equation*}

  Therefore, the induced emf is:
  \begin{equation*}
    \varepsilon_{induced} = - \frac{1}{2} (2.0 \; m)^2 (-0.870 \; T/s) = 1.74 \; V
  \end{equation*}

  The magnetic field is out of the page and decreasing so the induced emf is
  counterclockwise around the circuit, in the same direction
  as the emf of the battery. The total emf is:
  \begin{equation*}
    \varepsilon_{total} =
      \varepsilon_{induced} + \varepsilon_{battery} = 1.74 \; V + 20 \; V = 21.74 \; V
  \end{equation*}

\end{frame}

} % Worked example

% ------------------------------------------------------------------------------

%
% Worked example :
%

{
\problemslide

%
%
%

\begin{frame}{Worked example: Loop in non-uniform time-dependent $\vec{B}$}

  \begin{blockexmplque}{Question}

  \begin{minipage}[l]{0.34\textwidth}
   \begin{center}
  	 \includegraphics[width=0.88\textwidth]{./images/problems/lect08_square_loop_in_nonuniform_bfield}\\
   \end{center}
  \end{minipage}
  \begin{minipage}[r]{0.65\textwidth}
    As seen in the figure on the left, a square loop of wire has sides of
    length 2.0 cm. A magnetic field is directed out of the page; its magnitude
    is given by $B$ = 4.0$t^2$$y$, where $B$ is in Teslas, $t$ is in seconds,
    and $y$ is in meters.
    At $t$ = 2.5 s, what are the
    \begin{itemize}
      \item magnitude, and
      \item direction of the emf induced in the loop?
    \end{itemize}
  \end{minipage}

  \end{blockexmplque}

  Consider a (thin) strip of area of height $dy$ and width $\ell$ = 0.020 m.
  The strip is located at position $y$ (0 $<$ $y$ $<$ $\ell$).
  The magnetic field in that thin strip is uniform and, therefore,
  the magnetic flux through that strip is:
  \begin{equation*}
     d\Phi_{B} = B dA = (4 t^2 y)(\ell dy)
  \end{equation*}

\end{frame}

%
%
%

\begin{frame}{Worked example: Loop in non-uniform time-dependent $\vec{B}$}

  The total flux through the square loop is:
  \begin{equation*}
     \Phi_{B} = \int d\Phi_{B} = \int_{0}^{\ell} 4 t^2 y \ell dy
              = 4 t^2 \ell \int_{0}^{\ell} y dy
              = 2 t^2 \ell^3
  \end{equation*}
  Thus, Faraday's law yields:
  \begin{equation*}
    \varepsilon = - \frac{d\Phi_{B}}{dt} = - 4 t \ell^3
  \end{equation*}
  At $t$ = 2.5 s, the magnitude of the emf is:
  \begin{equation*}
    |\varepsilon| = 4 (2.5 \; s) (0.02 \; m)^3 = 8.0 \times 10^{-5} \; V
  \end{equation*}

  The emf direction is clockwise, by Lenz's law.

\end{frame}

} % Worked example

% ------------------------------------------------------------------------------

%
% Worked example :
%

{
\problemslide

%
%
%

\begin{frame}{Worked example: Charging a parallel-plate capacitor}

  \begin{blockexmplque}{Question}

    A parallel-plate capacitor has square plates of edge length $L$ = 1.0 m.
    A current of 2.0 A charges the capacitor, producing a uniform electric
    field $\vec{E}$ between the plates, with $\vec{E}$ perpendicular to the plates.\\
    \vspace{0.2cm}
    \begin{minipage}[l]{0.19\textwidth}
     \begin{center}
    	 \includegraphics[width=0.98\textwidth]{./images/problems/lect08_parallel_plate_capacitor_edge_view}\\
     \end{center}
    \end{minipage}
    \begin{minipage}[l]{0.19\textwidth}
     \begin{center}
    	 \includegraphics[width=0.98\textwidth]{./images/problems/lect08_parallel_plate_capacitor_top_view}\\
     \end{center}
    \end{minipage}
    \begin{minipage}[r]{0.60\textwidth}
      \begin{itemize}
        \item What is the displacement current $I_{d}$
          through the region between the plates?
        \item What is $dE/dt$ in this region?
        \item What is the displacement current encircled
          by the square dashed path of edge length $d$ = 0.50 m?
        \item What is $\oint \vec{B} \cdot d\vec{\ell}$
          around this square dashed path?
      \end{itemize}
    \end{minipage}

  \end{blockexmplque}

\end{frame}

%
%
%

\begin{frame}{Worked example: Charging a parallel-plate capacitor}

  As the current $I$ charges the capacitor, the electric field between
  the plates of the capacitor is changing.
  This produces a displacement current $I_{d}$ bet- ween the plates,
  which is given by:
  \begin{equation*}
    I_d = \epsilon_0 \frac{d\Phi_{E}}{dt}
  \end{equation*}

  Let $A$ be the area of a plate, $d$ the plate separation,
  and $E$ the magnitude of the electric field between the plates.
  $E$ is uniform, and it is given by:
  \begin{equation*}
    E = \frac{V}{d}
  \end{equation*}
  where $V$ is the potential difference across the plates.

  The current into the positive plate of the capacitor is
  \begin{equation*}
    I = \frac{dQ}{dt} = \frac{d}{dt}(CV) = C \frac{dV}{dt}
      = \frac{\epsilon_0 A}{d} \frac{d(Ed)}{dt}
      = \epsilon_0 A \frac{dE}{dt} = \epsilon_0 \frac{d\Phi_{E}}{dt} = I_{d}
  \end{equation*}

  At any time, the conduction current $I$ in the wires equals
  the displacement current $I_{d}$ in the gap between the plates,
  and thus $I_{d}$ = 2.0 A.

\end{frame}

%
%
%

\begin{frame}{Worked example: Charging a parallel-plate capacitor}

  The rate of change of the electic field is:
  \begin{equation*}
    I_d = \epsilon_0 A \frac{dE}{dt} \Rightarrow
    \frac{dE}{dt} = \frac{I_d}{\epsilon_0 A} \Rightarrow
  \end{equation*}
  \begin{equation*}
    \frac{dE}{dt} = \frac{2.0 \; A}{(8.85 \times 10^{-12} \; F/m)(1.0 \; m^2)}
                  = 2/3 \times 10^{11} \frac{V}{m \cdot s}
  \end{equation*}

  The displacement current $I_d^\prime$ through the indicated path is
  \begin{equation*}
    I_d^\prime = I_d \frac{d^2}{L^2}
               = (2.0 \; A) \frac{(0.5 \; m)^2}{(1.0 \; m)^2}
               = (2.0 \; A) \frac{1}{4} = 0.5 \; A
  \end{equation*}

   From Ampere's law, the integral of the magnetic field around the indicated path is
   \begin{equation*}
      \oint \vec{B} \cdot d\vec{\ell}
         = \mu_0 I_d^\prime
         = (1.26 \times 10^{-6} \; T \cdot m / A) (0.5 \; A)
         = 6.3 \times 10^{-7} \; T \cdot m
   \end{equation*}

\end{frame}

} % Worked example


% ------------------------------------------------------------------------------
% ------------------------------------------------------------------------------
