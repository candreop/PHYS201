
\renewcommand{\summarizedlecture}{3 }

%
%
%

\begin{frame}{Lecture \summarizedlecture - \lecturesummarytitle}

\begin{itemize}
\item {\bf To bring together a collection of charges I need to do work}
      (for example, In case of two like-sign charges I need to exert a force against the action of the field)
      \begin{equation*}
        W = \int \vec{F} d\vec{\ell}
      \end{equation*}
\item The work done can be positive or negative.
\item The work done is {\bf path-independent}
  \begin{itemize}
     \item I do the same work regardless of the path followed to bring the charges in their positions.
     \item We say that the electric force is {\bf conservative}.
  \end{itemize}
\item The work done is converted to {\bf electric potential energy}
\item We calculated the  potential energy for systems of 2, 3 and N charges as well as continuous distributions of charge.

\end{itemize}

\end{frame}

%
%
%

\begin{frame}{Lecture \summarizedlecture - \lecturesummarytitle (cont'd)}

For a
\begin{itemize}
\item system of 2 charges:
 \begin{equation*}
      U = \frac{q_1 q_2}{4\pi\epsilon_0} \frac{1}{|\vec{r}_{12}|}
 \end{equation*}

\item system of 3 charges:
 \begin{equation*}
   U = \frac{q_1 q_2}{4\pi\epsilon_0} \frac{1}{|\vec{r}_{12}|} +
       \frac{q_1 q_3}{4\pi\epsilon_0} \frac{1}{|\vec{r}_{13}|} +
       \frac{q_2 q_3}{4\pi\epsilon_0} \frac{1}{|\vec{r}_{23}|}
 \end{equation*}

\item system of N charges:
 \begin{equation*}
   U = \frac{1}{2} \sum_{i,j=1;i{\ne}j}^{N} \frac{q_i q_j}{4\pi\epsilon_0|\vec{r}_{ij}|}
 \end{equation*}

\item continuous charge distribution (with density $\rho$ over a volume $\tau$):
 \begin{equation*}
    U = \frac{1}{2} \int_{\tau} d\tau \int_{\tau^{\prime}} d\tau^{\prime}
        \frac{\rho(\vec{r}) \rho(\vec{r^{\prime}})}{4\pi\epsilon_0|\vec{r} - \vec{r^{\prime}}|}
 \end{equation*}

\end{itemize}

\end{frame}

%
%
%

\begin{frame}{Lecture \summarizedlecture - \lecturesummarytitle (cont'd)}

\begin{itemize}
  \item
  The electric {\bf potential energy is stored in the electric field}:
  \begin{equation*}
     U = \frac{\epsilon_0}{2} \int_{all\;space} |\vec{E}(\vec{r})|^2  d\tau
  \end{equation*}

  \item
  {\bf Relationship between force and potential energy}:\\
  \begin{equation*}
     U = \int \vec{F} d\vec{\ell}
  \end{equation*}
  \begin{equation*}
     \vec{F} = -\vec{\nabla}U
  \end{equation*}

  \item
  {\bf Electric potential (V)}: A scalar field
      \begin{itemize}
          \item It is the amount of work required to bring a charge Q at position $\vec{r}$, divided by the charge Q.
          \item SI units: Volts (V)
            \begin{itemize}
               \item Derived unit: One Joule per Coulomb
            \end{itemize}
      \end{itemize}
\end{itemize}

\end{frame}

%
%
%

\begin{frame}{Lecture \summarizedlecture - \lecturesummarytitle (cont'd)}

\begin{itemize}
  \item We also studied the {\bf circuital law for Electrostatics}
     \begin{itemize}
        \item Our second set of Maxwell's equations.
     \end{itemize}
  \vspace{0.2cm}
  \item Maxwell's equation we know so far:
\end{itemize}

\begin{center}
 {
  \begin{table}[H]
    \begin{tabular}{|l|c|c|}
      \hline
          & {\it Integral form} & {\it Differential form} \\
      \hline
      {\bf Gauss's law} &
        $\oint \vec{E} d\vec{S} = Q_{enclosed} / \epsilon_0$ &
        $\vec{\nabla} \cdot \vec{E} = \rho / \epsilon_0$ \\

      {\bf Circuital law} &
        $\oint \vec{E} d\vec{\ell} = 0$ &
        $\vec{\nabla} \times \vec{E} = 0$ \\
      \hline
    \end{tabular}
  \end{table}
 }
\end{center}

\begin{itemize}
  \item Because (in Electrostatics) the electric field has no rotation
        it can be expressed as the gradient of a scalar function (the electric potential):
        \begin{equation*}
           \vec{E} = - \vec{\nabla} V
        \end{equation*}
\end{itemize}


\end{frame}

%
%
%

\begin{frame}{Lecture \summarizedlecture - \lecturesummarytitle (cont'd)}

\begin{itemize}

  \item Need both the divergence and the curl of $\vec{E}$ (both Gauss' and circuital laws)
        to determine all three components of $\vec{E}$.
    \begin{itemize}
    {
      \item Gauss' and circuital laws provide a coupled system of 1$^{st}$ order p.d.e's.
    }
    \end{itemize}

  \item I can combine the Gauss' and circuital laws into a single 2$^{nd}$ order p.d.e for the
        electric potential V: {\bf Poisson equation}
        \begin{equation*}
           \vec{\nabla}^{2} V = - \frac{\rho}{\epsilon_0}
        \end{equation*}

  \item Away from sources ($\rho$=0) Poisson's equation becomes $\vec{\nabla}^{2} V = 0$
        which is known as the {\bf Laplace equation}.

  \item Using the Poisson (or Laplace) equations one can determine V (and, thus, $\vec{E}$)
        only using the appropriate {\bf boundary conditions}.

  \item {\bf A uniqueness theorem} is a proof that a given set of boundary conditions is sufficient.

\end{itemize}

\end{frame}
