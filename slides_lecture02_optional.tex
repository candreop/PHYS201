%
% Optional reading
%

\begin{frame}[plain,c]
\begin{center}
{\Huge \bf Optional reading for Lecture \thislecture}
\end{center}
\end{frame}

{
\programmingslide

%
%
%

\begin{frame}{PHYS201 scientific programming task for Lecture \thislecture}

{\small

If you did the previous task, you already have a program to calculate the
electric field (in 2-D) for an arbitrary distribution of discrete charges.\\
\vspace{0.2cm}

Generalize your previous program:
\begin{itemize}
{
  \item Move from a 2-D to a {\bf 3-D calculation}.
  \item Add an option to {\bf specify a continuous distribution of charge}
        (i.e. work with a user-defined charge density function $\rho(\vec{r})$)
}
\end{itemize}

\vspace{0.2cm}
What you will be doing, is to perform the following numerical integration:
\begin{equation*}
   \vec{E}(\vec{r}) = \frac{1}{4\pi\epsilon_0} \int_{\tau}
      d\tau^{\prime} \frac{\rho({\pvec{r}'})}{|\vec{r}-\pvec{r}'|^{3}} (\vec{r}-\pvec{r}')
\end{equation*}

\vspace{0.3cm}

Can you test Gauss' law numerically?

}
\end{frame}

%
%
%

\begin{frame}{PHYS201 scientific programming task for Lecture \thislecture}

{\small

As an example, use the following charge density in spherical coordinates:
\begin{equation*}
   \rho =
     \begin{cases}
       \frac{\rho_0}{(r/r_0)^2} e^{-r/r_0} cos^2\phi, & \text{if $r < 5 r_0$} \\
       & \\
       0, \text{otherwise}
     \end{cases}
\end{equation*}
where $\rho_0$ = 0.16 C/m$^{3}$ and r$_0$ = 10 cm.\\
%
% at r = 5 *r0, the charge contained is 6.24 * \rho_0 * ro^3
% 0.16 is 1/6.24
%

\vspace{0.2cm}
Calculate numerically the amount of charge Q enclosed in a sphere of radius r, as a function or r:\\
\begin{equation*}
  Q(r) = \int_{0}^{r} \int_{4\pi} d\tau \rho(\vec{r^\prime})
\end{equation*}

\vspace{0.2cm}
Confirm that your distribution plateaus to a value of $Q_{tot}$ for
r $>5r_0$, as the sphere encloses all regions of non-zero charge density.
What is the value of $Q_{tot}$?\\

\vspace{0.2cm}
Calculate the electric flux through the surface of a sphere with radius r = 5$r_0$
and confirm that:
\begin{equation*}
  \epsilon_0 \oint \vec{E} \cdot d\vec{S} = Q_{tot}
\end{equation*}
}

\end{frame}

} % programming
